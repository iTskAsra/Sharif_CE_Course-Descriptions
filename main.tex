\documentclass[12pt]{article}
\usepackage{amsmath}
\usepackage{amsfonts}
\usepackage{graphicx}
\usepackage{hyperref}
\usepackage[utf8]{inputenc}
\usepackage{geometry}
\geometry{a4paper}

\title{\vspace{3cm} Course Descriptions}
\author{B.Sc of Computer Engineering at Sharif University of Technology}
\date{\vspace{-3ex}}
\begin{document}

\maketitle

\section*{Introduction}
This document contains the descriptions of the courses offered in the Bachelor of Science degree in Computer Engineering at Sharif University of Technology. The source of the course descriptions is \href{https://docs.ce.sharif.edu}{docs.ce.sharif.edu}, which is available only in Persian. It is recommended that the auto-translate feature of a web browser (Edge or Chrome) be used for easy understanding of the content, if the viewer intends to verify this document themselves.

\noindent This document includes the content of 64 courses. The courses are brought in no particular order.

\noindent This document also contains Graduate level courses that are popular among Bachelor students at Sharif's CE Department. It is noteworthy that a Graduate level course does count towards the progression of the B.Sc. degree.
\newpage
\tableofcontents

\newpage




\section{System-on-Chip Design}
\subsection*{Course Code: 40757, Units: 3, Level: Master's, Prerequisite: None}

\textbf{Course Summary:} 
\begin{quote}
   With the advancement of integrated circuit technology, it is now possible to integrate various components of a digital system, including processors, memory, digital and analog blocks, and communication blocks, onto a single chip. This is referred to as System-on-Chip (SoC). In this course, students will learn how to design SoCs and will cover important concepts such as co-design of hardware and software, platform-based design, multi-processor SoCs, on-chip interconnection networks, and testing of SoCs.
\end{quote}

\textbf{Course Outline:}
\begin{itemize}
    \item Introduction to SoC architecture and design topics
    \item Co-design of hardware and software
    \item Design for platform-based systems
    \item Multi-processor SoCs (MPSoC)
    \item On-chip interconnection networks
    \item SoC testing (digital logic cores, embedded memory, analog cores, mixed-signal testing)
\end{itemize}

\textbf{Books Used:}
\begin{itemize}
    \item Michael J. Flynn, Wayne Luk, \textit{Computer System Design: System-on-Chip}, John Wiley \& Sons, Inc., 2011.
    \item Laung-Terng Wang, Charles E. Stroud, Nur A. Touba, \textit{System-on-Chip Test Architectures}, Morgan Kaufmann Publishers, 2008.
    \item Natalie Enright Jerger, Tushar Krishna, and Li-Shiuan Peh, \textit{On-Chip Networks}, Second Edition, Morgan \& Claypool Publishers, 2017.
\end{itemize}

\textbf{Evaluation:}
\begin{itemize}
    \item Theoretical Exercises: 3 points
    \item Mid-term and Final Exams: 15 points
    \item Quizzes: 2 points
\end{itemize}


\newpage

\section{Embedded Systems}
\subsection*{Course Code: 40462, Units: 3, Prerequisite: Computer Architecture}

\textbf{Course Summary:} 
\begin{quote}
   An embedded system is a computer system designed to perform specific tasks within a larger system, which is typically non-computational. Statistics show that the majority of computers worldwide (over 80\%) are embedded systems. Embedded systems are also fundamental to concepts in computer engineering, such as cyber-physical systems and the Internet of Things (IoT). This course introduces the design and analysis of embedded systems, with a focus on microcontrollers, system architectures, hardware and software structure, and programming techniques for embedded applications.
\end{quote}

\textbf{Course Outline:}
\begin{itemize}
    \item Introduction to Embedded Systems (1 session)
    \begin{itemize}
        \item Overview of embedded systems, their importance, and applications
        \item Reactive systems, sensors, and actuators
    \end{itemize}
    \item Microcontrollers as Key Elements of Embedded Systems (2 sessions)
    \begin{itemize}
        \item Introduction to microcontrollers, their architecture, and importance
        \item Analog-to-digital conversion, ports, and serial communication
    \end{itemize}
    \item Resource and Task Management in Microcontrollers (3 sessions)
    \begin{itemize}
        \item Software implementation of embedded systems
        \item Role of infinite loops, remote debugging, and emulators
    \end{itemize}
    \item Common Hardware Platforms for Embedded Systems (6 sessions)
    \begin{itemize}
        \item Overview of Arduino and Raspberry Pi as common embedded platforms
        \item Comparison and application areas of these platforms
    \end{itemize}
    \item Automata-Based Programming (4 sessions)
    \begin{itemize}
        \item Introduction to automata-based programming for reactive systems
        \item Hierarchical structures and Mealy/Moore descriptions
    \end{itemize}
    \item StateCharts Language (3 sessions)
    \begin{itemize}
        \item MoC in StateCharts and hierarchical design
        \item Event and reaction descriptions, use of timers, and real-time concerns
    \end{itemize}
    \item Real-Time, Energy Consumption, and Reliability in Embedded Systems (4 sessions)
    \begin{itemize}
        \item Methods for reducing energy consumption and improving reliability
        \item Trade-offs between real-time performance, energy consumption, and reliability
    \end{itemize}
    \item Introduction to the Internet of Things and Embedded Systems Role (7 sessions)
    \begin{itemize}
        \item IoT definitions, applications, and architectures
        \item Communications challenges, D2D communication, 5G, and energy concerns in IoT
    \end{itemize}
    
\end{itemize}

\textbf{Books Used:}
\begin{itemize}
    \item Peter Marwedel, \textit{Embedded System Design}, 1st Edition, Springer, 2006.
    \item Ahmad Kardaan and Seyed Amir Asghari, \textit{Applications of Embedded Systems in Measurement and Control}, Kian Rayaneh Sabz, 2008.
    \item Adrian McEwen and Hakim Cassimally, \textit{Designing the Internet of Things}, 1st Edition, Wiley, 2013.
    \item Online documents on Raspberry Pi and Arduino.
\end{itemize}

\textbf{Evaluation:}
\begin{itemize}
    \item Exercises: 3 points
    \item Project: 2 points
    \item Mid-term and Final Exams: 13 points
    \item Quizzes: 2 bonus points
\end{itemize}


\newpage

\section{Real-Time Systems}
\subsection*{Course Code: 40453, Units: 3, Prerequisite: None, Co-requisite: Operating Systems}

\textbf{Course Summary:} 
\begin{quote}
   The goal of this course is to introduce students to the concepts of real-time systems and timely task execution. The students will learn to design and analyze systems that not only compute the correct results but also perform tasks on time despite the presence of various periodic and non-periodic tasks. Understanding the conditions for real-time feasibility, proper task scheduling, and prioritization is also a key objective of this course.
\end{quote}

\textbf{Course Outline:}
\begin{itemize}
    \item Introduction to Real-Time Systems, Classifications, and Applications
    \begin{itemize}
        \item Motivation, definitions, types of tasks (soft vs. hard, periodic vs. aperiodic), components of a typical real-time system
    \end{itemize}
    \item Modeling and Verification of Real-Time System Characteristics (using Petri Nets)
    \begin{itemize}
        \item Concurrent processing, resource sharing, liveness and boundedness, deadline and timing constraints, execution time estimation and analysis
    \end{itemize}
    \item Periodic Task Scheduling on a Single Processor
    \begin{itemize}
        \item Static and dynamic priority scheduling (Rate Monotonic, EDF, etc.)
        \item Schedulability analysis criteria
    \end{itemize}
    \item Non-preemptive vs. Preemptive Tasks
    \item Scheduling of Aperiodic and Sporadic Tasks in Combination with Periodic Tasks on a Single Processor
    \begin{itemize}
        \item Schedulability conditions and task distribution methods
    \end{itemize}
    \item Scheduling Algorithms (FCFS, Polling Server, Deferred Server, Slack Stealing, Sporadic Servers, etc.)
    \item Brief Overview of Multi-Processor Task Scheduling
    \item Reliability, Availability, and Fault Tolerance in Real-Time Systems
    \item Real-Time Communications and Protocols
    \begin{itemize}
        \item Time constraints in communication and real-time protocols in networks
    \end{itemize}
    
\end{itemize}

\textbf{Books Used:}
\begin{itemize}
    \item G. C. Buttazzo, \textit{Hard Real-Time Computing Systems: Predictable Scheduling Algorithms and Applications}, 3rd Edition, Springer, 2011.
    \item J. W. S. Liu, \textit{Real-Time Systems}, Prentice Hall, 2000.
    \item Ph. A. Laplante, \textit{Real-Time Systems Design and Analysis}, 3rd Edition, IEEE Press \& Wiley InterScience, 2004.
    \item C. M. Krishna and Kang G. Shin, \textit{Real-Time Systems}, McGraw-Hill, 1997.
    \item Some real-time related conference and journal papers.
\end{itemize}

\textbf{Evaluation:}
\begin{itemize}
    \item Theoretical Exercises: 2 points
    \item Mid-term and Final Exams: 15 points
    \item Practical Project: 3 points
\end{itemize}

\newpage

\section{General Mathematics 1 and 2}
\subsection*{Course Code: 22015/22016, Units: 4/4, Prerequisites: None/General Mathematics 1}

\subsubsection*{Course Objectives}
\begin{enumerate}
    \item Introduce students to differential and integral calculus as tools for solving problems, especially nonlinear ones.
    \item Present concepts of n-dimensional linear algebra as a foundation for analyzing problems with \( n \) parameters.
    \item Help students understand the principle of approximation and motivate computational solutions using calculators and computers.
    \item Emphasize concepts and intuition while avoiding reliance on computational techniques that are easily performed by calculators or computers.
    \item Avoid abstract concepts that lack motivation while focusing on providing a conceptual framework and essential tools for problem formulation and resolution.
    \item Introduce the concept of differential equations and systems of differential equations naturally throughout the course, with applications such as growth, decay, oscillatory motions, and linear/nonlinear transformations.
    \item Structure the topics based on educational goals to ensure content feels distinct from high school material and remains engaging.
    \item Include topics that serve as prerequisites for Differential Equations and Engineering Mathematics, reducing the content load of those future courses.
    \item Present the curriculum as a single-year course to allow flexibility in teaching both courses across academic years.
\end{enumerate}

\subsubsection*{Course Topics}
\begin{enumerate}
    \item \textbf{Numbers:} Historical overview of numbers, rational and irrational numbers, completeness axiom, complex numbers, and some applications; sequences and series of numbers.
    \item \textbf{Single-Variable Functions:} Limits and continuity; properties of continuous functions on closed intervals; differentiability, linear approximation, applications of derivatives, Taylor polynomials, and their applications.
    \item \textbf{Single-Variable Integration:} Definite and indefinite integrals, fundamental theorems, transcendental functions, differential equations, approximation methods, traditional applications of integration, including an introduction to probability.
    \item \textbf{Differential Equations:} Growth and decay problems, oscillatory motions.
    \item \textbf{Functional Series:} Power series, Taylor series, Fourier series, and their applications, including solving differential equations using power series.
    \item \textbf{Introduction to n-Dimensional Linear Algebra:} Properties of linear spaces (\( \mathbb{R}^n \)), inner product and applications, subspaces, linear mappings and applications, concepts of volume and determinants, diagonalization of symmetric matrices.
    \item \textbf{Curves in Plane and Space:} Concepts of curvature and torsion; fundamental theorems.
    \item \textbf{Functions from \( \mathbb{R}^n \) to \( \mathbb{R}^n \):} General properties, representations of multivariable functions, concepts of limits, continuity, and partial derivatives.
    \item \textbf{Multivariable Differentiation:} Differentiability, gradients, chain rule, higher-order derivatives, multivariable Taylor polynomials and series, inverse function theorem, and implicit function theorem.
    \item \textbf{Optimization:} Critical and regular points, classification of critical points, finding maxima and minima with and without constraints, Lagrange multipliers.
    \item \textbf{Multiple Integration:} Basic concepts, computation, improper integrals, general variable transformation formula.
    \item \textbf{Line Integrals and Vector Fields:} Basic concepts and applications, computation, conservative fields, and potentials.
    \item \textbf{Surface Integrals on Curved Surfaces:} Analysis of smooth parametric and general surfaces, computation of surface integrals, and applications.
    \item \textbf{Vector Analysis:} Concepts of divergence and curl with geometric and physical interpretations, Green's theorem, Stokes' theorem, and divergence theorem in various forms, with applications in scalar and vector potential problems.
\end{enumerate}

\newpage

\section{Differential Equations}
\subsection*{Course Code: 22034, Units: 3, Prerequisite: General Mathematics 2 (or concurrent enrollment)}

\subsubsection*{Course Objectives}
\begin{enumerate}
    \item Emphasize modeling and studying mathematical models of physical, natural, and social systems.
    \item Study differential equations using analytical, geometric, and qualitative methods.
    \item Focus on conceptual understanding and intuition while avoiding reliance on computational techniques that can easily be performed using calculators or computers.
    \item Use mathematical software to solve differential equations.
\end{enumerate}

\subsubsection*{Course Topics}
\begin{enumerate}
    \item Solving ordinary differential equations (ODEs) using analytical, geometric, and qualitative methods.
    \item Linear ODEs, especially second-order equations, linear independence of solutions, method of undetermined coefficients, and variation of parameters.
    \item Systems of linear equations and the method of undetermined coefficients.
    \item Nonlinear autonomous equations, singular points, stability, and asymptotic stability.
    \item Lyapunov's second method.
    \item Predator-prey problems.
    \item Fourier series.
    \item Partial differential equations (PDEs) of second order: heat equation, wave equation, and Laplace's equation.
\end{enumerate}
\newpage

\section{Computer Architecture Lab}
\subsection*{Course Code: 40103, Units: 1, Prerequisite: Computer Architecture, Logic Circuits Lab}

\subsubsection*{Course Objectives}
The primary objective of this course is to familiarize students with practical methods for implementing key components of a computer architecture (such as the arithmetic-logic unit, control unit, and memory). The course aims to provide students with a practical understanding of designing and implementing an instruction set architecture on a target architecture.

\subsubsection*{Course Topics}
\begin{enumerate}
    \item Introduction to CAD tools for designing and testing the correctness of logic circuits.
    \item Introduction to a sample simulator (e.g., Quartus).
    \item Design, implementation, and testing of a sample circuit (e.g., a circuit for adding two two-digit decimal numbers) using a simulator.
    \item Design and implementation of computational architectures.
    \item Design and implementation of a 4-bit fixed-point multiplier.
    \item Design and implementation of a floating-point adder/subtractor.
    \item Design and implementation of a decimal-to-binary converter.
    \item Design and implementation of a simple computer architecture.
    \item Design and implementation of a calculation unit with selectable source.
    \item Design and implementation of a calculation unit with program-controlled operations.
    \item Complete design and implementation of a computer with data memory and jump instructions.
    \item Design and implementation of a processor.
    \item Design and implementation of a micro-programmed control circuit.
    \item Performance testing of the implemented circuit.
\end{enumerate}

\subsubsection*{References}
\begin{enumerate}
    \item D. Patterson and J. L. Hennessy. \textit{Computer Organization \& Design, The Hardware/Software Interface}, 4th Edition, Morgan Kaufmann Publishing, 2011.
    \item M. Mano. \textit{Computer System Architecture}, 3rd Edition, Prentice Hall, 1992.
\end{enumerate}

\newpage

\section{Discrete Structures}
\subsection*{Course Code: 40115, Units: 3, Prerequisite: None}

\subsubsection*{Course Objectives}
This course aims to familiarize students with the concepts, structures, and techniques of discrete mathematics widely used in computer science and engineering. The goals include developing foundational skills such as understanding and constructing precise mathematical proofs, creative problem-solving, understanding basic results in logic, combinatorics, number theory, graph theory, and computation theory, and providing the mathematical prerequisites necessary for many other courses in various computer engineering disciplines.

\subsubsection*{Course Topics}
\begin{enumerate}
    \item \textbf{Logic (3 sessions)} 
    \begin{itemize}
        \item Basics of logic, propositions, equivalent propositions
        \item Predicates, quantifiers, inference rules
        \item Methods of proof
    \end{itemize}
    \item \textbf{Set Theory and Functions (2 sessions)}
    \begin{itemize}
        \item Fundamentals of set theory, set operators, countable and uncountable sets
        \item One-to-one and onto functions, function composition, inverse functions, sequences
    \end{itemize}
    \item \textbf{Number Theory (2 sessions)}
    \begin{itemize}
        \item Divisibility, congruence, modular arithmetic
        \item Prime numbers, Euler's theorem, introduction to cryptography
    \end{itemize}
    \item \textbf{Induction (2 sessions)}
    \begin{itemize}
        \item Mathematical induction, well-ordering principle
        \item Strong induction, structural induction
    \end{itemize}
    \item \textbf{Counting (4 sessions)}
    \begin{itemize}
        \item Basic counting principles, permutations, and combinations
        \item Binomial coefficients, permutations, and combinations with repetition
        \item Inclusion-exclusion principle, distributing objects into boxes
        \item Pigeonhole principle
    \end{itemize}
    \item \textbf{Discrete Probability (2 sessions)}
    \begin{itemize}
        \item Probability theory, probability distribution functions, conditional probabilities
        \item Random variables, expected value, variance
    \end{itemize}
    \item \textbf{Recurrence Relations (3 sessions)}
    \begin{itemize}
        \item Recurrence problems
        \item Solving recurrence relations (homogeneous and non-homogeneous)
        \item Generating functions
    \end{itemize}
    \item \textbf{Relations (2 sessions)}
    \begin{itemize}
        \item Relations and their properties, representation of relations, composition of relations
        \item Equivalence relations, closures
    \end{itemize}
    \item \textbf{Partial Orders and Boolean Algebra (2 sessions)}
    \begin{itemize}
        \item Partially ordered sets, Hasse diagrams, topological sorting
        \item Lattices, Boolean algebra, properties of Boolean algebra
    \end{itemize}
    \item \textbf{Graphs (3 sessions)}
    \begin{itemize}
        \item Basic definitions, special graphs, bipartite graphs, graph representation, graph isomorphism
        \item Paths and connectivity, Eulerian and Hamiltonian paths
        \item Planar graphs, Euler's formula, graph coloring
    \end{itemize}
    \item \textbf{Trees (1 session)}
    \begin{itemize}
        \item Trees and forests, special trees, rooted trees, spanning trees
    \end{itemize}
    \item \textbf{Algebraic Structures (1 session, optional)}
    \begin{itemize}
        \item Monoids, rings, groups, Abelian groups
    \end{itemize}
    \item \textbf{Modeling Computation (3 sessions)}
    \begin{itemize}
        \item Languages and grammars, finite state machines
        \item Language recognition, regular languages
        \item (Optional) Turing machines
    \end{itemize}
\end{enumerate}

\subsubsection*{Evaluation}
\begin{itemize}
    \item Theory Assignments: 15\%
    \item Exams (Midterm, Final, and Quizzes): 85\%
\end{itemize}

\subsubsection*{References}
\begin{enumerate}
    \item K. H. Rosen. \textit{Discrete Mathematics and Its Applications}. 8th Edition, McGraw Hill, 2018.
    \item R. P. Grimaldi. \textit{Discrete and Combinatorial Mathematics: An Applied Introduction}. 5th Edition, Pearson Addison Wesley, 2004.
    \item A. Engel. \textit{Problem-Solving Strategies}. Springer, 1998.
\end{enumerate}

\newpage

\section{Fundamentals of Electrical and Electronic Circuits}
\subsection*{Course Code: 40124, Units: 3, Prerequisite: Physics II}

\subsubsection*{Course Objectives}
This course aims to introduce students to electrical components and methods for analyzing electrical circuits in both the time and Laplace domains. It also provides an introduction to electronic circuits that constitute logic gates in several widely used technologies.

\subsubsection*{Course Topics}
\begin{enumerate}
    \item \textbf{Introduction to Electrical Circuits, Basic Elements, and Their Relations (8 sessions)}
    \begin{itemize}
        \item Kirchhoff's voltage and current laws
        \item Series and parallel connections of resistive elements
        \item Circuit analysis methods: Node and mesh analysis
        \item Linearity and the superposition principle
        \item Thevenin and Norton equivalent circuits
        \item Operational amplifiers and practical applications
    \end{itemize}
    \item \textbf{Circuit Analysis in the Time Domain (5 sessions)}
    \begin{itemize}
        \item Introduction to waveforms (step, pulse, impulse, sinusoidal)
        \item Introduction to energy storage elements and active elements
        \item First-order electrical circuits
        \item Second-order electrical circuits
    \end{itemize}
    \item \textbf{Circuit Analysis in the Frequency Domain (5 sessions)}
    \begin{itemize}
        \item Laplace transform
        \item Impedance and admittance
        \item Circuit analysis using the Laplace transform
    \end{itemize}
    \item \textbf{Diodes and Transistors (2 sessions)}
    \begin{itemize}
        \item Diode characteristics, models, and applications
        \item Logic inverter
        \item General characteristics and models of transistors
    \end{itemize}
    \item \textbf{Field-Effect Transistors (6 sessions)}
    \begin{itemize}
        \item Structure, operation, and characteristics of enhancement-mode MOSFETs
        \item Types of inverters using transistors
        \item Pass transistors and transmission gates
        \item Static CMOS logic
    \end{itemize}
    \item \textbf{Practical Circuits (4 sessions)}
    \begin{itemize}
        \item Latches and flip-flops in static CMOS logic
        \item Shift registers
        \item Types of RAM and ROM memory
        \item Analog-to-digital converters (ADC)
        \item Digital-to-analog converters (DAC)
    \end{itemize}
\end{enumerate}

\subsubsection*{Evaluation}
\begin{itemize}
    \item Theory Assignments: 3 points
    \item Midterm and Final Exams: 14 points
    \item Quizzes: 3 points
\end{itemize}

\subsubsection*{References}
\begin{enumerate}
    \item William H. Hayt \& Jack E. Kemmerly. \textit{Engineering Circuit Analysis}. 7th Edition, McGraw-Hill, 2007.
    \item Ernest Kuh \& Charles Desoer, \textit{Basic Circuit Theory}, Translated by Dr. Jabedar Maralani, University of Tehran Publications, 2016 (1395).
    \item Adel Sedra \& Kenneth Smith, \textit{Microelectronic Circuits}, Translated by Majid Malekan \& Haleh Vahedi, 4th Edition, Daneshgahi Sciences Press, 2002 (1381).
    \item Mahmoud Tabandeh, \textit{Pulse Techniques and Digital Circuits}, Sharif University Scientific Publishing Institute, 1997 (1376).
\end{enumerate}

\newpage

\section{Computer Structure and Machine Language}
\subsection*{Course Code: 40126, Units: 3, Prerequisites: Foundations of Programming, Logic Circuits}

\subsubsection*{Course Objectives}
The primary objective of this course is to familiarize students with the components of a computer and their interactions during the execution of program instructions. Programming in machine and assembly languages and their translation enable students to gain a deeper understanding of instruction set architectures and efficient utilization of machines. By the end of the course, students will be prepared to learn how to design and implement these components in the Computer Architecture course.

\subsubsection*{Course Topics}
\begin{enumerate}
    \item \textbf{Computer History}
    \begin{itemize}
        \item Introduction to computer generations and types
        \item Von Neumann model
    \end{itemize}
    \item \textbf{Data Representation}
    \begin{itemize}
        \item Numbers: integer/fractional, signed/unsigned, fixed-point/floating-point, binary/decimal, etc.
        \item Characters: 7-bit and 8-bit base codes, 16-bit and 32-bit universal codes
    \end{itemize}
    \item \textbf{Computer Structure}
    \begin{itemize}
        \item Central Processing Unit (CPU), Arithmetic Logic Unit (ALU), Registers, Control Unit (CU), Main Memory
        \item Shared Bus, Fetch-Execute Cycle
        \item Addressing Modes: Immediate, Direct, Indirect, Relative, Implicit, Indexed, Segmented, Paged
    \end{itemize}
    \item \textbf{Assembly Language Programming and Translation to Machine Language on Simple Computers}
    \begin{itemize}
        \item Assembler, Debugger, Compiler, Linker, Loader
        \item Overview of the Instruction Set of at least one CISC computer (e.g., Intel 8086, IBM 360/370, or MC68000 recommended)
        \item Introduction to the computer structure and addressing methods
        \item Instruction set and assembly programming for the selected computer
        \item Structured programming constructs (subroutines, macros, etc.)
        \item Interrupts and their management
    \end{itemize}
    \item \textbf{RISC Computer Overview}
    \begin{itemize}
        \item Overview of the Instruction Set of at least one RISC computer (e.g., MIPS recommended)
        \item Introduction to the computer structure and addressing methods
        \item Instruction set and assembly programming for the selected computer
        \item Structured programming constructs (subroutines, macros, etc.)
        \item Interrupts and their management
    \end{itemize}
\end{enumerate}

\subsubsection*{Evaluation}
\begin{itemize}
    \item Theory Assignments: 3 points
    \item Midterm and Final Exams: 15 points
    \item Quizzes: 2 points
\end{itemize}

\subsubsection*{References}
\begin{enumerate}
    \item J. L. Antonakos. \textit{The 68000 Microprocessor: Hardware and Software Principles and Applications}. Prentice Hall, 2004.
    \item M. A. Mazidi et al. \textit{The x86 PC: Assembly Language, Design, and Interfacing}. Prentice Hall, 2010.
    \item G. Struble. \textit{Assembler Language Programming: The IBM System/360}. Addison-Wesley, 1971.
    \item D. A. Patterson and J. L. Hennessy. \textit{Computer Organization and Design MIPS Edition: The Hardware/Software Interface}. 5th Edition, Elsevier (Morgan Kaufmann), 2013.
\end{enumerate}

\newpage

\section{Engineering Probability and Statistics}
\subsection*{Course Code: 40181, Units: 3, Prerequisite: Calculus I}

\subsubsection*{Course Objectives}
This course aims to introduce students to fundamental concepts of probability theory and statistical inference and their applications in computer engineering, such as data modeling problems like regression. These concepts include interpretations and axioms of probability, single and multivariable probability distributions, conditional probability and statistical independence, random variables and expectations, functions defined on random variables, the exponential family of distributions, the Central Limit Theorem, the Law of Large Numbers, and hypothesis testing.

\subsubsection*{Course Topics}
\begin{enumerate}
    \item \textbf{Interpretations of Probability}
    \begin{itemize}
        \item Axioms of Probability
        \item Operations on Events
        \item Statistical Independence, Conditional Probability, and Bayes' Rule
    \end{itemize}
    \item \textbf{Random Variables}
    \begin{itemize}
        \item Expectation and Its Properties
        \item Higher-Order Moments and Characteristic Functions
        \item Functions of a Single Random Variable
    \end{itemize}
    \item \textbf{Joint Probability Distributions}
    \begin{itemize}
        \item Joint Probability Density Function and Bayes' Law
        \item Conjugate Prior Distributions
        \item Exponential Family of Distributions
        \item Special Distributions
        \item Joint Moments
        \item Functions of Two or More Random Variables
    \end{itemize}
    \item \textbf{Key Theorems}
    \begin{itemize}
        \item Central Limit Theorem
        \item Law of Large Numbers
        \item Convergence in Probability
    \end{itemize}
    \item \textbf{Estimation and Inference}
    \begin{itemize}
        \item Maximum Likelihood and Maximum A Posteriori Probability Estimators
        \item Bayesian Estimation
        \item Properties of Estimators
    \end{itemize}
    \item \textbf{Statistical and Hypothesis Testing}
\end{enumerate}

\subsubsection*{Evaluation}
\begin{itemize}
    \item Theory Assignments: 4 points
    \item Midterm Exam 1: 4 points
    \item Midterm Exam 2: 4 points
    \item Final Exam: 6 points
    \item Quizzes: 2 points
\end{itemize}

\subsubsection*{References}
\begin{enumerate}
    \item A. Poppulis and S. Pillai. \textit{Probability, Random Variables and Stochastic Processes}. 4th Edition, McGraw Hill, 2002 (Chapters 1 through 8).
    \item S. Ross. \textit{A First Course in Probability}. 10th Edition, Prentice Hall, 2019.
    \item G. Casella and R. L. Berger. \textit{Statistical Inference}. 2nd Edition, Wadsworth Press, 2002.
\end{enumerate}

\newpage

\section{Digital Systems Design Lab}
\subsection*{Course Code: 40203, Units: 1, Prerequisite: Logic Circuits Lab, Corequisite: Digital Systems Design}

\subsubsection*{Course Objectives}
The objective of this laboratory is to provide students with hands-on experience in digital systems design using automatic digital design tools (CAD Tools) and the implementation of digital systems using programmable elements such as CPLD and FPGA.

\subsubsection*{Lab Topics}
\begin{enumerate}
    \item \textbf{Experiment 1:} Designing Combinational Circuits Using Schematic Tools
    \item \textbf{Experiment 2:} Designing Sequential Circuits Using Schematic Tools
    \item \textbf{Experiment 3:} Data Flow Description
    \item \textbf{Experiment 4:} Behavioral Description
    \item \textbf{Experiment 5:} Multiplier Design
    \item \textbf{Experiment 6:} Incubator Design
    \item \textbf{Experiment 7:} UART Design
    \item \textbf{Experiment 8:} Complex Number ALU Design
    \item \textbf{Experiment 9:} Implementation of Tri-state Buffers
    \item \textbf{Experiment 10:} Simple Processor Implementation
\end{enumerate}

\subsubsection*{References}
\begin{enumerate}
    \item S. Palnitkar. \textit{Verilog HDL: A Guide to Digital Design and Synthesis}. 2nd Edition, Prentice Hall, 2003.
    \item ACEX 1K Programmable Logic Family Data Sheet. Available at www.altera.com.
    \item ModelSim User's Manual. Available at www.actel.com.
    \item Introduction to the Quartus II Software. Available at www.altera.com.
\end{enumerate}

\newpage

\section{Logic Design Lab}
\subsection*{Course Code: 40206, Units: 1, Prerequisite: Logic Circuits, Corequisite: None}

\subsubsection*{Course Objectives}
The objective of this lab is to familiarize students with the implementation of logic circuits, including shift registers, adders, subtractors, counters, latches, and data buses. The Logic Design Lab provides practical experience with the theories learned in the Logic Circuits course.

\subsubsection*{Lab Topics}
\begin{enumerate}
    \item Familiarization with laboratory equipment and guides.
    \item Understanding transfer characteristics and fan-out in TTL chips.
    \item Familiarization with sequential circuits.
    \item Timing circuits.
    \item Shift registers.
    \item Counters.
    \item Design and implementation of finite state machines.
    \item Implementation of a timer for a washing machine.
    \item Implementation of a telephone system.
    \item Familiarization with components of a simple computer.
    \item Understanding the arithmetic and logic unit (ALU), registers, and buses.
    \item Implementation of a hardware stack.
    \item Design of a simple binary computer.
\end{enumerate}

\subsubsection*{References}
\begin{enumerate}
    \item D. Patterson and J. L. Hennessy. \textit{Computer Organization \& Design, The Hardware/Software Interface}. 4th Edition, Morgan Kaufmann Publishing, 2011.
    \item M. Mano. \textit{Computer System Architecture}. 3rd Edition, Prentice Hall, 1992.
\end{enumerate}

\newpage

\section{Logic Design}
\subsection*{Course Code: 40212, Units: 3, Prerequisite: None, Corequisite: None}

\subsubsection*{Course Objectives}
The aim of this course is to introduce students to logic gates as circuit realizations of logical operators and simple integrated circuits constructed with a limited number of gates. In this course, students will learn methods for analyzing and designing combinational and synchronous sequential circuits. They will also become familiar with the structure, operation, and application of some simple integrated circuits, which form the building blocks of more complex systems, providing a foundation for understanding hardware components such as processors.

\subsubsection*{Course Topics}
\begin{enumerate}
    \item \textbf{Representation of Numbers in Base 2} (2 sessions)
        \begin{itemize}
            \item Number base conversion
            \item Representation of negative numbers using sign magnitude, 2's complement, and 1's complement
            \item Addition and subtraction of numbers
            \item Carry bits
            \item BCD number representation
        \end{itemize}
    
    \item \textbf{Combinational Circuits} (3 sessions)
        \begin{itemize}
            \item Boolean algebra and key algebra
            \item Logic gates
            \item Minterms and Maxterms
            \item SOP and POS forms
            \item Delay and critical path
        \end{itemize}
    
    \item \textbf{Simplification of Logical Functions} (4 sessions)
        \begin{itemize}
            \item Algebraic methods
            \item Karnaugh maps
            \item Concepts of don't care and forbidden inputs
            \item Implementation of two-level circuits
            \item Race, Hazard, and Glitch concepts
            \item Hazard elimination
        \end{itemize}

    \item \textbf{Combinational Components} (5 sessions)
        \begin{itemize}
            \item Decoder and multiplexer
            \item Implementing functions with decoders and multiplexers
            \item Encoder and priority encoder
            \item Demultiplexer
            \item Half adder and full adder
            \item Ripple-carry adder and carry-lookahead adder
            \item Comparator
            \item Read-Only Memory (ROM)
        \end{itemize}

    \item \textbf{Multivalue Logic} (2 sessions)
        \begin{itemize}
            \item Three-state and four-state logic
            \item Tri-state gates
            \item Open-collector gates
            \item Wired logic
            \item Pull-up and pull-down resistors
        \end{itemize}

    \item \textbf{Sequential Circuits} (4 sessions)
        \begin{itemize}
            \item Concept of sequential circuits
            \item Types of Latches and forbidden inputs in Latches
            \item Level-sensitive flip-flops, edge-sensitive flip-flops, and Master-Slave flip-flops
            \item Synchronous and asynchronous reset inputs
            \item Setup-time and Hold-time
        \end{itemize}

    \item \textbf{Finite State Machines (FSM)} (4 sessions)
        \begin{itemize}
            \item Mathematical concept of FSM
            \item State diagrams
            \item State tables
            \item Excitation tables
            \item Steps for implementing FSM
            \item Mealy and Moore models and their differences
        \end{itemize}

    \item \textbf{Sequential Components} (4 sessions)
        \begin{itemize}
            \item Registers and shift registers
            \item Universal registers
            \item Synchronous counters
            \item Johnson counters
            \item Asynchronous counters (Ripple counters)
        \end{itemize}

    \item \textbf{Programmable Logic Devices (PLDs)} (2 sessions)
        \begin{itemize}
            \item Introduction to PLDs
            \item Introduction to SPLDs
            \item PAL and PLA and comparison with ROM
            \item PAL with tri-state logic
        \end{itemize}
\end{enumerate}

\subsubsection*{References}
\begin{enumerate}
    \item M. Morris Mano. \textit{Digital Design}. 5th Edition, Prentice Hall, 2006.
    \item Victor P. Nelson, H. Troy Nagle, Bill D. Carroll, David Irwin. \textit{Digital Logic Circuit Analysis and Design}. Prentice Hall, 1995.
    \item Franklin P. Prosser and David E. Winkel. \textit{The Art of Digital Design: An Introduction to Top-Down Design}. Prentice Hall, 1987.
    \item Alireza Ejlali. \textit{Digital Circuits}, 1st Edition, Nasir Publishing, 1397 (Persian).
\end{enumerate}

\newpage

\section{Numerical Computations}
\subsection*{Course Code: 40215, Units: 3, Prerequisite: Differential Equations, Corequisite: None}

\subsubsection*{Course Objectives}
This course aims to familiarize undergraduate students with numerical methods for solving problems in various fields of science and engineering. These methods can approximate solutions to scientific problems where exact calculations are either impossible with conventional mathematical methods or computationally impractical due to complexity. In some cases, exact solutions can be computed but are highly complex, introducing errors. Numerical methods provide approximate solutions with limited error and reduced complexity. At the start of this course, students will be introduced to the concepts of error, and later, they will learn different numerical methods for solving engineering problems. Additionally, using efficient software environments to solve problems, comparing numerical methods, and graphically presenting results to summarize and conclude are also key objectives. The course will also include real-world problems whose solutions are challenging with traditional mathematical methods but are solvable using numerical methods.

\subsubsection*{Course Topics}
\begin{enumerate}
    \item \textbf{Introduction to Software Tools} (2 sessions)
        \begin{itemize}
            \item MATLAB or Python usage
            \item Matrix calculations
            \item Vectors and plotting
            \item Files and function definitions
            \item Built-in functions in the chosen tool
        \end{itemize}
    
    \item \textbf{Errors} (4 sessions)
        \begin{itemize}
            \item Introduction to errors
            \item Floating-point representation
            \item Sources of errors
            \item Relative and absolute errors
            \item Rounding, intrinsic, and truncation errors
            \item Error propagation and graphical process of error propagation
            \item Numerical instability
        \end{itemize}

    \item \textbf{Solving Nonlinear Equations} (4 sessions)
        \begin{itemize}
            \item Introduction to finding roots of nonlinear single-variable functions
            \item Bisection method
            \item Fixed-point iteration method
            \item Secant method
            \item Newton-Raphson method
            \item Simple iteration or fixed-point method
            \item Convergence rates of different methods
            \item Necessary/sufficient conditions for convergence in Newton-Raphson, secant, and simple iteration methods
            \item Horner's method for polynomial evaluation
            \item Generalized Newton-Raphson method for solving systems of nonlinear equations
            \item Intuitive and mathematical proofs for the above methods
        \end{itemize}

    \item \textbf{Interpolation, Extrapolation, and Curve Fitting} (5 sessions)
        \begin{itemize}
            \item Introduction to interpolation, extrapolation, and curve fitting
            \item Various interpolation methods including Lagrange's method, Newton's divided difference method, forward, backward, and central differences
            \item Proof and error analysis for these methods
            \item Polynomial curve fitting using the least squares method
            \item Curve fitting using linearization techniques
            \item Extrapolation
        \end{itemize}

    \item \textbf{Numerical Integration and Differentiation} (4 sessions)
        \begin{itemize}
            \item Introduction to numerical integration and differentiation
            \item Various methods of numerical integration including rectangle method, midpoint rule, trapezoidal rule, Gaussian quadrature, Simpson’s 1/3 and 3/8 rules, and Romberg's method
            \item Error analysis of the above methods
            \item Mathematical and intuitive proofs for the methods
            \item Numerical differentiation using methods such as the midpoint rule, central difference, and three-point methods
            \item Error order analysis for the methods and using Richardson extrapolation to improve results in numerical differentiation
        \end{itemize}

    \item \textbf{Solving Initial Value Problems for Ordinary Differential Equations (ODEs)} (4 sessions)
        \begin{itemize}
            \item Introduction to ODEs
            \item Single-step methods including Taylor series method, Euler’s method, modified Euler’s method, second-order Runge-Kutta methods (Heun, midpoint, and modified Euler), third-order and fourth-order Runge-Kutta methods
            \item Multistep methods like Adams-Moulton method
            \item Error analysis and comparison of the above methods
            \item Converting higher-order differential equations into systems of linear differential equations
            \item Converting single-step methods for solving linear ODEs into numerical methods suitable for solving systems of linear ODEs
        \end{itemize}

    \item \textbf{Numerical Solution of Linear Systems of Equations} (4 sessions)
        \begin{itemize}
            \item Introduction to solving systems of linear equations
            \item Introduction to matrices
            \item Direct methods for solving linear systems including matrix inversion, Cramer's rule, Gaussian elimination (forward, backward, and Gauss-Jordan elimination), and LU decomposition (Cholesky, Doolittle, and Crout decompositions)
            \item Iterative methods such as Jacobi method and Gauss-Seidel method
            \item Eigenvalues and eigenvectors, power method for estimating the dominant eigenvalue and corresponding eigenvector, and Gershgorin circle theorem
        \end{itemize}
\end{enumerate}

\subsubsection*{Evaluation}
\begin{itemize}
    \item \textbf{Exams}: Midterm and final exams (50\% of total grade)
    \item \textbf{Assignments}: 6 theoretical assignments (solving scientific problems using numerical methods), 6 practical assignments using the introduced software tool (40\% of total grade)
    \item \textbf{Project}: Project topic selected with the help of the instructor. The project can be research-oriented or practical (using the introduced tool). The research project must be reported in a report. Practical projects are preferred for gaining proficiency with the introduced tool. Group projects, if defined clearly, can have a significant positive impact on students' teamwork abilities (10\% of total grade)
\end{itemize}

\subsubsection*{References}
\begin{enumerate}
    \item S. Pal. \textit{Numerical Methods Principles, Analysis and Algorithms}. Oxford University Press, 2010.
    \item J. Kiusalaas. \textit{Numerical Methods in Engineering with Python 3}. Cambridge University Press, 2013.
    \item J. Kiusalaas. \textit{Numerical Methods in Engineering with MATLAB}. Cambridge University Press, 2015.
    \item C. B. Moler. \textit{Numerical Computing with MATLAB}. MathWorks, 2013.
\end{enumerate}

\newpage

\section{Digital Systems Design}
\subsection*{Course Code: 40223, Units: 3, Prerequisite: Computer Organization and Language, Corequisite: None}

\subsubsection*{Course Objectives}
This course aims to introduce students to the Verilog hardware description language (HDL), teaching them how to design hardware using HDLs at various levels of abstraction. It also focuses on the internal structure of programmable circuits and how hardware circuits can be implemented in FPGAs and CPLDs.

\subsubsection*{Course Topics}
\begin{enumerate}
    \item \textbf{ASM and FSM}
        \begin{itemize}
            \item FSM diagrams and their applications
            \item Modeling and synthesizing control circuits using FSM
            \item ASM diagrams and designing digital systems using them
            \item Control units and data paths
            \item Synthesizing data paths from ASM diagrams
            \item Methods of synthesizing control units from ASM diagrams
        \end{itemize}
    
    \item \textbf{Introduction to Hardware Description Languages (HDL)}
        \begin{itemize}
            \item Overview of HDL languages
            \item Key features and differences between concurrent and sequential HDL codes
            \item The importance of using HDL languages
        \end{itemize}
    
    \item \textbf{Verilog Language Introduction}
        \begin{itemize}
            \item Overview of Verilog language features
            \item Comparison between Verilog and other HDLs
            \item Reasons for using HDL languages in digital design
            \item Basics of Verilog syntax and structure
            \item Top-down and bottom-up design approaches
            \item Structural and behavioral modeling
            \item Introduction to modules, initial, always, and @ in Verilog
            \item Modular design and Verilog features for modularity
        \end{itemize}
    
    \item \textbf{Data Types and Basic Concepts in Verilog}
        \begin{itemize}
            \item Wire and register types and their differences
            \item Four-value logic and signal power concepts
            \item Array and vector concepts and their applications
            \item Real, Integer, Time, and String data types
            \item Parameter and parameterized design in Verilog
            \item System tasks, directives, and macros in Verilog
            \item Hierarchical naming and its applications
        \end{itemize}
    
    \item \textbf{Structural Modeling in Verilog}
        \begin{itemize}
            \item Ports in modules and types of ports
            \item Mapping ports and connection rules
            \item Gate-level design in Verilog
            \item Delay modeling in structural description
            \item Dataflow modeling in Verilog
            \item Inertial vs. transport delay modeling
        \end{itemize}

    \item \textbf{Behavioral Modeling in Verilog}
        \begin{itemize}
            \item Blocking and non-blocking assignments in behavioral modeling
            \item Event control in behavioral modeling
            \item Control structures such as decision-making and loops
            \item Functions and tasks in Verilog
            \item Types of event control (level-sensitive, edge-sensitive, named)
            \item Timing control techniques
            \item Modeling inertial and transport delays in behavioral modeling
            \item Race conditions in concurrent statements
        \end{itemize}

    \item \textbf{Verilog Code Synthesis and Simulation}
        \begin{itemize}
            \item Writing synthesizable Verilog code
            \item Rules for synthesizable code (avoiding delays, division operators, etc.)
            \item Methods to avoid combinational loops
            \item Impact of loops in behavioral descriptions on synthesis
            \item Three-state logic and its impact on synthesis
        \end{itemize}

    \item \textbf{Digital System Design with PLDs}
        \begin{itemize}
            \item Features of digital systems
            \item Levels of abstraction and modeling techniques
            \item Applications of programmable logic devices (PLDs)
            \item Overview of SPLDs and CPLDs
            \item Structure of SPLDs (PAL, PLA, ROM)
            \item Structure of CPLDs
            \item Technologies for SPLD and CPLD fabrication
            \item Case studies of CPLDs
            \item FPGAs and their structure
            \item Types of FPGAs (LUT-based and MUX-based)
            \item FPGA technologies (Anti-fuse and SRAM-based)
            \item Methods for constructing programmable connections in FPGAs
            \item Case studies of FPGA technologies
            \item Latest FPGA products and FPSoCs
        \end{itemize}
\end{enumerate}

\subsubsection*{Evaluation}
\begin{itemize}
    \item \textbf{Theoretical Exercises}: 3 points
    \item \textbf{Midterm and Final Exams}: 15 points
    \item \textbf{Quizzes}: 2 points
\end{itemize}

\subsubsection*{References}
\begin{enumerate}
    \item Samir Palnitkar. \textit{Verilog HDL: A Guide to Digital Design and Synthesis}. 2nd Edition, SunSoft Press, 2003.
    \item S. Brown, J. Rose. \textit{FPGA and CPLD Architectures: A Tutorial}. IEEE Design and Test of Computers, pp. 42-57, 1996.
    \item Altera Data Sheets. Available at \texttt{www.altera.com}.
    \item Xilinx Data Sheets. Available at \texttt{www.xilinx.com}.
    \item Actel Data Sheets. Available at \texttt{www.actel.com}.
\end{enumerate}

\newpage

\section{Advanced Programming}
\subsection*{Course Code: 40244, Units: 3, Prerequisite: Fundamentals of Programming, Corequisite: None}

\subsubsection*{Course Objectives}
This course introduces object-oriented programming concepts using the Java programming language. The intrinsic features of Java, its programming capabilities, and the differences in approach between Java and other similar languages are emphasized. Topics such as concurrent programming and software quality are also addressed. It is assumed that students have prior knowledge of at least one programming language and are familiar with problem-solving techniques, such as searching, sorting, and mathematical operations. The focus of this course is on object-oriented concepts.

\subsubsection*{Course Topics}
\begin{enumerate}
    \item \textbf{Introduction to Java (1 session)}
        \begin{itemize}
            \item History of Java
            \item Features of the Java language
            \item Writing the first Java program
        \end{itemize}
    
    \item \textbf{Basic Programming Concepts in Java (3 sessions)}
        \begin{itemize}
            \item Variables, methods, conditionals, loops
            \item Primitive data types
            \item Strings and arrays
        \end{itemize}
    
    \item \textbf{Introduction to Object-Oriented Design and Programming (2 sessions)}
        \begin{itemize}
            \item History and evolution of programming paradigms to object-oriented programming
            \item Basic concepts of object-oriented programming
            \item Object-oriented design mindset
            \item Encapsulation, interfaces, classes, packages, and access modifiers
        \end{itemize}
    
    \item \textbf{Object-Oriented Programming in Java (6 sessions)}
        \begin{itemize}
            \item Defining classes in Java
            \item Objects in memory and memory management
            \item Object initialization and destruction
            \item Garbage collector in Java
            \item Passing parameters in Java
            \item Constructors and the \texttt{this} keyword
            \item Static members and the package concept
            \item Introduction to UML class diagrams
        \end{itemize}
    
    \item \textbf{Inheritance (2 sessions)}
        \begin{itemize}
            \item Concept of inheritance
            \item Access modifiers: \texttt{protected}, \texttt{abstract}, \texttt{super}
            \item Multiple inheritance in other languages
        \end{itemize}
    
    \item \textbf{Polymorphism (1 session)}
        \begin{itemize}
            \item Polymorphism using inheritance
            \item The concept of \texttt{virtual} in C++
            \item \texttt{final} members in Java
        \end{itemize}
    
    \item \textbf{Interfaces (1 session)}
        \begin{itemize}
            \item Using interfaces
            \item Multiple inheritance using interfaces
        \end{itemize}
    
    \item \textbf{Software Testing (1 session)}
        \begin{itemize}
            \item Concept of software quality assurance and its importance
            \item Unit testing and writing unit tests using JUnit
            \item Testing exceptions using JUnit
            \item Introduction to mocking and a mocking library in Java
        \end{itemize}
    
    \item \textbf{Design Patterns (1 session)}
        \begin{itemize}
            \item Definition and significance of design patterns in software engineering
            \item GoF design patterns and their categories
            \item Detailed explanation of Singleton, State, Strategy, Observer, and Facade patterns
            \item Introduction to MVC architectural pattern
        \end{itemize}
    
    \item \textbf{Refactoring (2 sessions)}
        \begin{itemize}
            \item What is refactoring and its importance for clean code
            \item Signs of bad code
            \item Refactoring patterns
            \item Creating methods, transferring features between objects, organizing data
            \item Simplifying conditional expressions, method calls, and error management
        \end{itemize}
    
    \item \textbf{Error and Exception Handling (2 sessions)}
        \begin{itemize}
            \item Traditional error management model
            \item Java's error management framework
            \item Benefits of this model
            \item \texttt{finally}, runtime exceptions
        \end{itemize}
    
    \item \textbf{Generics (1 session)}
        \begin{itemize}
            \item Generic methods and classes
            \item Applications of generics
            \item Creating and using generic classes
            \item Generics and inheritance
            \item Difference between Java generics and C++ templates
        \end{itemize}
    
    \item \textbf{Collections and Containers (2 sessions)}
        \begin{itemize}
            \item Data structures available in Java
            \item Collections, ArrayList, LinkedList, Set, Map
            \item Using Iterator
        \end{itemize}
    
    \item \textbf{File, Streams, and Networking (2 sessions)}
        \begin{itemize}
            \item File I/O in Java
            \item Serialization
            \item Socket programming
        \end{itemize}
    
    \item \textbf{Concurrency (1 session)}
        \begin{itemize}
            \item The need for concurrency
            \item Concurrency in Java
            \item Life cycle of a thread
            \item Introduction to synchronization and critical sections
        \end{itemize}
    
    \item \textbf{Reflection (1 session)}
        \begin{itemize}
            \item Need for runtime type information (RTTI)
            \item RTTI in Java and its applications
        \end{itemize}
    
    \item \textbf{Advanced Concepts (1 session)}
        \begin{itemize}
            \item Inner classes and anonymous classes
            \item Annotations
            \item Enumeration
        \end{itemize}
\end{enumerate}

\subsubsection*{Evaluation}
\begin{itemize}
    \item \textbf{Exams (Midterm, Final, and Quizzes)}: 50\%
    \item \textbf{Programming Assignments}: 25\%
    \item \textbf{Project (3 phases throughout the term)}: 25\%
\end{itemize}

\subsubsection*{References}
\begin{enumerate}
    \item P. Deitel, H. Deitel. \textit{Java: How to Program}. 11th Edition, Pearson Education, 2017.
    \item B. Eckel. \textit{Thinking in Java}. 4th Edition, Prentice Hall, 2006.
    \item M. Fowler, K. Beck, J. Brant, W. Opdyke, D. Roberts. \textit{Refactoring: Improving the Design of Existing Code}. Addison-Wesley, 1999.
\end{enumerate}

\newpage

\section{Data Structures and Algorithms}
\subsection*{Course Code: 40254, Units: 3, Prerequisite: Discrete Structures, Corequisite: Advanced Programming}

\subsubsection*{Course Objectives}
In this course, students will become familiar with methods of algorithm analysis, simple and moderately advanced but important data structures, and some basic algorithms. Emphasis will be placed on analyzing and proving the correctness of algorithms. The course assumes prior knowledge of programming in either C++ or Java and familiarity with recursive problem-solving techniques. The algorithms are taught independently of the language and according to the reference textbook.

\subsubsection*{Course Topics}
\begin{enumerate}
    \item \textbf{Introduction (1 session)}
        \begin{itemize}
            \item Levels of abstraction
            \item Problem-solving steps and abstraction
            \item Data models, data types, data structures, abstract data types, objects
        \end{itemize}
    
    \item \textbf{Algorithm Analysis (3 sessions)}
        \begin{itemize}
            \item Time complexity analysis: insertion sort
            \item Growth of functions
            \item Divide-and-conquer analysis
        \end{itemize}
    
    \item \textbf{Divide and Conquer (2 sessions)}
        \begin{itemize}
            \item Merge sort, counting inversions, longest common subsequence, matrix multiplication
            \item Master theorem for divide and conquer recurrences
        \end{itemize}
    
    \item \textbf{Randomized Algorithms (1 session)}
        \begin{itemize}
            \item Approximate median calculation, hiring problem
        \end{itemize}
    
    \item \textbf{Basic Data Structures (1 session)}
        \begin{itemize}
            \item Queue and stack
            \item Linked list
        \end{itemize}
    
    \item \textbf{Tree Data Structures (5 sessions)}
        \begin{itemize}
            \item Different implementations of trees, tree traversal, structural induction
            \item Expression trees, converting between different notations of a mathematical expression
            \item Trie data structure
            \item Binary search tree
            \item Priority queue (min-heap and max-heap)
        \end{itemize}
    
    \item \textbf{Sorting (4 sessions)}
        \begin{itemize}
            \item Decision tree and lower bounds
            \item Heap sort
            \item Quick sort (randomized analysis)
            \item Optimal number of comparisons in sorting
            \item Linear time sorting: counting, radix, bucket sort
            \item External sorting (optional)
        \end{itemize}
    
    \item \textbf{Order Statistics (2 sessions)}
        \begin{itemize}
            \item Finding minimum and maximum
            \item Selecting the k-th smallest element (randomized and deterministic algorithms)
        \end{itemize}
    
    \item \textbf{Hashing (2 sessions)}
        \begin{itemize}
            \item Chaining in hashing
            \item Open addressing in hashing
            \item Perfect hashing
        \end{itemize}
    
    \item \textbf{Advanced Data Structures (3 sessions)}
        \begin{itemize}
            \item Disjoint sets
            \item Balanced binary trees: Red-Black trees
            \item Interval tree
        \end{itemize}
    
    \item \textbf{Graphs (3 sessions)}
        \begin{itemize}
            \item Different graph representations
            \item Depth-first search (DFS) and breadth-first search (BFS) and their applications
            \item Topological sort, strongly connected components
            \item Shortest path algorithms: Dijkstra’s and Bellman-Ford algorithms
        \end{itemize}
\end{enumerate}

\subsubsection*{Evaluation}
\begin{itemize}
    \item \textbf{Five exercise sets}: Each set includes several theoretical problems and programming tasks. Theoretical problems do not need to be submitted.
    \item \textbf{Five short quizzes}: Based on theoretical problems above + a similar problem (3 points)
    \item \textbf{Five practical assignments}: (3 points)
    \item \textbf{Midterm exam}: (6 points)
    \item \textbf{Final exam}: (8 points)
\end{itemize}

\subsubsection*{References}
\begin{enumerate}
    \item Mohammad Qadsi, \textit{Data Structures and Foundations of Algorithms}, 4th Edition, Fatemi Publications, 2014.
    \item Mohammad Qadsi and Aidin Nasiri Shargh, \textit{600 Multiple Choice Questions on Data Structures and Algorithms}, 6th Edition, Fatemi Publications, 2018.
    \item T. Cormen, C. Leiserson, R. Rivest, and C. Stein, \textit{Introduction to Algorithms}, 3rd Edition, MIT Press, 2011.
\end{enumerate}

\newpage

\section{Linear Algebra}
\subsection*{Course Code: 40282, Units: 3, Prerequisite: Calculus II}

\subsubsection*{Course Objectives}
The goal of this course is to introduce students to the fundamental concepts of linear algebra and how to apply and implement them in an appropriate software environment. Familiarity with the concepts of this course enables the analysis of linear transformations and systems using matrices, operators, and related definitions. Optimization problems, as one of the most common applications of linear algebra, will also be explored.

\subsubsection*{Course Topics}
\begin{enumerate}
    \item \textbf{Vector Spaces}
        \begin{itemize}
            \item Linear transformations and matrices
            \item Vector space of linear transformations
            \item Algebraic structure of linear transformations
        \end{itemize}
    
    \item \textbf{Matrices and Rank}
        \begin{itemize}
            \item Inverse of linear transformations
            \item Duality
        \end{itemize}
    
    \item \textbf{Linear Systems}
        \begin{itemize}
            \item Volume and determinant
        \end{itemize}
    
    \item \textbf{Polynomials}
        \begin{itemize}
            \item Zeros of polynomials
            \item Factorization of polynomials in complex and real fields
        \end{itemize}
    
    \item \textbf{Eigenvalues and Eigenvectors}
        \begin{itemize}
            \item Subspaces of invariants
            \item Eigenvectors and eigenvalues
            \item Linearly independent eigenvectors
        \end{itemize}
    
    \item \textbf{Inner Product Spaces}
        \begin{itemize}
            \item Inner product and distance definition
            \item Orthogonal bases
            \item Operators in inner product spaces
        \end{itemize}
    
    \item \textbf{Operators and Decompositions}
        \begin{itemize}
            \item Polar decomposition
            \item Singular value decomposition (SVD)
            \item Cholesky decomposition
            \item LU decomposition
            \item QR decomposition
            \item Adjoints and normal operators
            \item Isometric and isometric operators
            \item Positive operators
        \end{itemize}
\end{enumerate}

\subsubsection*{Evaluation}
\begin{itemize}
    \item \textbf{Assignments}: 6 points
    \item \textbf{Two midterm exams}: 8 points
    \item \textbf{Final exam}: 6 points
    \item \textbf{Quizzes}: 1 point
\end{itemize}

\subsubsection*{References}
\begin{enumerate}
    \item Sheldon Axler, \textit{Linear Algebra}, Springer, 2015.
    \item Gilbert Strang, \textit{Linear Algebra and Its Applications}, 4th Edition, Cengage Learning, 2006.
    \item David Clay, \textit{Linear Algebra and Its Applications}, 4th Edition, Pearson, 2011.
\end{enumerate}

\newpage

\section{Computer Architecture}
\subsection*{Course Code: 40323, Units: 3, Prerequisite: Computer Organization and Language}

\subsubsection*{Course Objectives}
In the Computer Organization and Language course, students are introduced to the various components of a computer and how they interact to execute the instructions of a program. The primary objective of this course is to teach how to design and implement these components, and the various techniques used in implementing different architectures for different applications.

\subsubsection*{Course Topics}
\begin{enumerate}
    \item \textbf{Overview of Basic Components and History of Computers}
        \begin{itemize}
            \item Combinational and sequential circuits
            \item Advantages of digital technology over analog
            \item Multiplexers, decoders, tri-state gates, buses
        \end{itemize}
    
    \item \textbf{Abstraction Levels and Computer Description}
        \begin{itemize}
            \item History and generations of computers
        \end{itemize}
    
    \item \textbf{Number Representation}
        \begin{itemize}
            \item Different methods for digital representation of signed and unsigned numbers, integers and floating-point numbers
            \item Absolute and relative precision, representation ranges
        \end{itemize}
    
    \item \textbf{Processor and Computer Performance}
        \begin{itemize}
            \item Factors affecting computer performance
            \item Definition of performance (inverse of execution time)
            \item Performance formula
            \item Benchmarking and examples
        \end{itemize}
    
    \item \textbf{Designing the Execution Unit (Data Path) and Hardwired Control}
        \begin{itemize}
            \item Overview of addressing modes
            \item Transfer levels and languages between registers (RTL)
            \item Instruction Set Architecture (ISA)
            \item Step-by-step analysis and design of a sample processor (MIPS)
        \end{itemize}
    
    \item \textbf{Interrupt Implementation and Polling Method}
        \begin{itemize}
            \item Interrupt handling and methods
        \end{itemize}
    
    \item \textbf{Control Unit Description and Design}
        \begin{itemize}
            \item Microprogrammed control unit
            \item Comparison of advantages and disadvantages of microprogrammed control versus hardwired control
        \end{itemize}
    
    \item \textbf{Memory Systems}
        \begin{itemize}
            \item Overview of memory types and hierarchy
            \item Cache memory and different mapping methods (direct-mapped, fully associative, set-associative)
        \end{itemize}
    
    \item \textbf{Arithmetic Algorithms}
        \begin{itemize}
            \item Algorithms for addition, subtraction, multiplication, and division
            \item Booth's encoding and array multiplier for multiplication
        \end{itemize}
    
    \item \textbf{I/O Methods}
        \begin{itemize}
            \item Handshaking techniques
        \end{itemize}
    
    \item \textbf{Advanced Architectures}
        \begin{itemize}
            \item Overview of acceleration and parallelism methods
            \item Pipeline architecture and execution time considerations
        \end{itemize}
\end{enumerate}

\subsubsection*{Evaluation}
\begin{itemize}
    \item \textbf{Theoretical Assignments}: 3 points
    \item \textbf{Midterm and Final Exams}: 15 points
    \item \textbf{Quizzes}: 2 points
\end{itemize}

\subsubsection*{References}
\begin{enumerate}
    \item D. A. Patterson and J. L. Hennessey, \textit{Computer Organization and Design}, 3rd Edition, Elsevier (Morgan Kaufmann), 2005.
    \item M. Mano, \textit{Computer System Architecture}, 3rd Edition, Prentice Hall, 1992.
\end{enumerate}

\newpage

\section{Design of Programming Languages}
\subsection*{Course Code: 40364, Units: 3, Prerequisite: Advanced Programming}

\subsubsection*{Course Objectives}
The main objectives of this course are: 
\begin{enumerate}
    \item A review of the natural evolution of concepts and methods in the design and implementation of different generations of programming languages in an empirical and step-by-step approach.
    \item Introduction to engineering methods for designing and implementing programming languages, particularly focusing on domain-specific languages (DSLs) and the importance of their design and implementation.
    \item Familiarity with the implementation of interpreters, particularly in the context of virtual machines.
    \item A review of the principles and issues related to programming language design, along with the data structures used in implementing or realizing a programming environment.
\end{enumerate}

\subsubsection*{Course Topics}
\begin{enumerate}
    \item \textbf{Introduction}
        \begin{itemize}
            \item Evolutionary history of programming languages and introduction of some significant languages from a historical perspective.
            \item Comparative introduction of the major programming paradigms (imperative-procedural, object-oriented, rule-based programming, and declarative-functional programming) and their approach to the concept of a program.
        \end{itemize}
    
    \item \textbf{Interpretation vs. Compilation}
        \begin{itemize}
            \item Comparing interpretation and compilation from both language design and implementation perspectives.
        \end{itemize}
    
    \item \textbf{Language Engineering}
        \begin{itemize}
            \item Familiarity with existing tools for designing domain-specific languages and implementing efficient interpreters. Special focus: practical exercises with DrRacket.
        \end{itemize}
    
    \item \textbf{Functional Programming}
        \begin{itemize}
            \item Review of core functional programming concepts and lambda calculus, along with practical exercises and projects. Suggested languages: Scheme (based on syntax and semantics from Friedman’s book) or Racket (based on Krishnamurthi’s book).
            \item Modern interpretation of Lisp-based programming on programmable platforms.
            \item Optional introduction to functional programming features in Java 8 and above.
        \end{itemize}
    
    \item \textbf{Designing and Implementing a Programming Language Interpreter}
        \begin{itemize}
            \item Language with computational expressions (no side effects).
            \item Adding non-recursive and recursive procedures (subprograms) to the language and its interpreter.
            \item Adding concepts of variable scope and binding domains to the language and its interpreter.
            \item Adding memory manipulation (reference-type variables) to the language and its interpreter.
            \item Introducing type systems (type-annotated variables) in the language and its interpreter.
            \item Implementing modular and object-oriented programming features (modules, classes, and objects) in the designed language and its interpreter.
        \end{itemize}
    
    \item \textbf{Selected Advanced Topics}
        \begin{itemize}
            \item Overview of notable programming languages such as ML, Haskell, Scala, and F\#.
            \item Review of external factors influencing language design and implementation, such as requirements for parallel or concurrent programming, real-time systems, web-based programming, and component-based or service-oriented software engineering.
            \item Introduction to programming language semantics and reasoning based on them:
                \begin{itemize}
                    \item Operational semantics
                    \item Denotational semantics
                    \item Axiomatic semantics (Hoare logic)
                \end{itemize}
        \end{itemize}
\end{enumerate}

\subsubsection*{Evaluation}
\begin{itemize}
    \item \textbf{Midterm Exam}: 25\%
    \item \textbf{Final Exam}: 40\%
    \item \textbf{Assignments}: 20\%
        \begin{itemize}
            \item Functional programming exercises
            \item Step-by-step interpreter design exercises
            \item Theoretical assignments
        \end{itemize}
    \item \textbf{Project}: 15\%
\end{itemize}

\subsubsection*{References}
\begin{enumerate}
    \item D. P. Friedman, M. Wand, \textit{Essentials of Programming Languages}, 3rd Edition, MIT Press, 2008.
    \item S. Krishnamurthi, \textit{Programming Languages: Application and Interpretation}, 2nd Edition, 2017.
    \item M. Felleisen, R. B. Findler, M. Flatt, S. Krishnamurthi, E. Barzilay, J. McCarthy, S. Tobin-Hochstadt, \textit{A Programmable Programming Language}, Communications of the ACM, Vol. 61, No. 3, pp. 62-71, March 2018.
    \item Racket programming language and its toolkits.
\end{enumerate}

\newpage

\section{Database Design}
\subsection*{Course Code: 40384, Units: 3, Prerequisite: Data Structures and Algorithms}

\subsubsection*{Course Objectives}
In this course, students will become familiar with the concepts of data modeling and database design. By the end of the semester, students are expected to have a complete understanding of the topics outlined in the syllabus.

\subsubsection*{Course Topics}
\begin{enumerate}
    \item \textbf{Database Approach and Database Systems} (3 sessions)
        \begin{itemize}
            \item Introduction to the course
            \item Definition of databases
            \item File-based approach vs. database approach
            \item Components of a database environment
            \item Types of database system architectures (centralized, client-server, distributed)
            \item Components of relational DBMS (RDBMS, OLTP)
        \end{itemize}
    
    \item \textbf{Semantic Data Modeling using ER and EER} (4 sessions)
        \begin{itemize}
            \item Entity
            \item Attribute
            \item Relationship
            \item ER and EER diagrams
            \item Types of constraints
            \item Techniques such as allocation, generalization, decomposition, combination, and aggregation
            \item Characteristics of semantic modeling techniques
        \end{itemize}

    \item \textbf{Principles of Database Design} (2 sessions)
        \begin{itemize}
            \item Introduction to tabular structures and relational databases
            \item Top-down design methodology (transforming semantic models into logical designs)
        \end{itemize}
    
    \item \textbf{Introduction to SQL and Database Implementation} (3 sessions)
        \begin{itemize}
            \item Introduction to SQL language
            \item DDL and DCL commands
            \item DML commands
            \item SQL integration in programming languages
            \item Transaction implementation
            \item Parameterized queries
        \end{itemize}

    \item \textbf{Three-Tier Database Architecture} (3 sessions)
        \begin{itemize}
            \item ANSI three-tier architecture
            \item View (perceptual) layer
            \item Internal and external views
            \item Transformations between layers
            \item Types of indices (B-Tree, B+-Tree, and Hash)
            \item Operations at the external level and related challenges
            \item Physical and logical data independence
        \end{itemize}
    
    \item \textbf{Fundamental Concepts of Relational Data Model} (2 sessions)
        \begin{itemize}
            \item Relational model components
            \item Relation and related concepts
            \item Keys in the relational model
            \item Overview of relational database design principles
        \end{itemize}

    \item \textbf{Integrity Constraints in Relational Model} (2 sessions)
        \begin{itemize}
            \item General integrity constraints (C1, C2)
            \item User-defined integrity constraints
            \item Mechanisms for applying user constraints (Assertions and Triggers)
        \end{itemize}

    \item \textbf{Operations in Relational Database} (3 sessions)
        \begin{itemize}
            \item Relational algebra
            \item Relational calculus
        \end{itemize}

    \item \textbf{Theory of Dependencies and Normalization} (3 sessions)
        \begin{itemize}
            \item Dependency theory concepts
            \item Normal forms (up to BCNF; other levels for individual study)
            \item Optimal decomposition
        \end{itemize}

    \item \textbf{Database Security} (1 session)
        \begin{itemize}
            \item User management
            \item Authentication
            \item Access control
            \item Data encryption
        \end{itemize}

    \item \textbf{NoSQL Database Systems} (2 sessions)
        \begin{itemize}
            \item Reasons for using NoSQL databases
            \item CAP theorem
            \item Key-value, column-oriented, graph-oriented, and document-oriented NoSQL databases
        \end{itemize}

    \item \textbf{(Optional) Introduction to Data Warehousing} (1 session)
        \begin{itemize}
            \item Introduction to Data Warehouses
            \item OLAP concepts
        \end{itemize}
\end{enumerate}

\subsubsection*{Evaluation}
\begin{itemize}
    \item \textbf{Midterm Exam}: 30\%
    \item \textbf{Final Exam}: 35\%
    \item \textbf{Assignments}: 17\%
    \item \textbf{Project}: 13\%
    \item \textbf{Quizzes and Class Activities}: 5\%
\end{itemize}

\subsubsection*{References}
\begin{enumerate}
    \item Mohammad Taghi Rouhani Rankouhi, \textit{Fundamentals of Databases}, 4th Edition, 2011.
    \item R. Elmasri, S. Navathe, \textit{Fundamentals of Database Systems}, 8th Edition, Pearson, 2019.
    \item A. Silberschatz, H. F. Korth, S. Sudarshan, \textit{Database System Concepts}, 6th Edition, McGraw-Hill, 2010.
    \item C. J. Date, \textit{An Introduction to Database Systems}, 8th Edition, Pearson, 2003.
    \item T. Connolly, C. Begg, \textit{Database Systems}, 6th Edition, Pearson, 2014.
    \item R. Ramakrishnan, J. Gehrke, \textit{Database Management Systems}, 4th Edition, McGraw-Hill, 2014.
\end{enumerate}

\newpage

\section{Operating Systems Lab}
\subsection*{Course Code: 40408, Units: 1, Prerequisite: Operating Systems}

\subsubsection*{Course Objectives}
The objective of this lab is to teach the various components of the Linux operating system, how to utilize them, and implement algorithms for each of these components. After completing this lab, students will be familiar with the structure of the Linux operating system and will be capable of modifying and compiling it. The general topics of the lab are as follows, although the specifics of each experiment may vary from semester to semester. Basic experiments will always be covered, and then the focus will shift to various topics.

\subsubsection*{Course Topics}
\begin{enumerate}
    \item \textbf{Compiling and Installing Linux}
    \item \textbf{Programming with C++ and Shell in Linux}
    \item \textbf{Using Linux System Calls in Programs}
    \item \textbf{Examining Operating System Behavior (proc/ directory)}
    \item \textbf{Creating, Running, and Terminating Processes and Threads (using pthread library)}
    \item \textbf{Process and Thread Synchronization and Communication}
    \item \textbf{Memory Management, Shared Memory, and Virtual Memory}
    \item \textbf{CPU Scheduling}
    \item \textbf{Using Installable File Systems}
    \item \textbf{Disk Scheduling and I/O Scheduling}
    \item \textbf{Design and Implementation of Device Drivers}
    \item \textbf{Using Linux Security Mechanisms}
    \item \textbf{Introduction to Real-Time and Embedded Operating Systems}
    \item \textbf{Introduction to Windows Research Kernel}
\end{enumerate}

\subsubsection*{References}
\begin{enumerate}
    \item P. J. Salzman, M. Burian, and O. Pomerantz, \textit{The Linux Kernel Module Programming Guide}, 2007.
    \item K. Wall, M. Watson, and M. Whitis, \textit{Linux Programming Unleashed}, Macmillan Computer Publishing, 1999.
    \item M. Mitchell, J. Oldham, and A. Samuel, \textit{Advanced Linux Programming}, New Rivers, 2001.
    \item C. S. Rodriguez, G. Fischer, and S. Smolski, \textit{The Linux® Kernel Primer: A Top-Down Approach for x86 and PowerPC Architectures}, Prentice-Hall, 2005.
    \item J. Corbet, A. Rubini, and G. Kroah-Hartman, \textit{Linux Device Drivers}, O'Reilly Books, 2005.
\end{enumerate}

\newpage

\section{Compiler Design}
\subsection*{Course Code: 40414, Units: 3, Prerequisite: Data Structures and Algorithms}

\subsubsection*{Course Objectives}
Compiler design and construction is one of the fundamental concepts in computer science. Although the methods for building compilers are somewhat limited, they can be used for building interpreters and translators for a wide variety of languages and machines. In this course, the subject of compiler construction is introduced through the description of the main components of a compiler, their duties, and interactions. After an introduction to the components of a compiler and types of grammars, various translation phases, such as lexical, syntactic, and semantic analysis, as well as code generation and optimization, are discussed.

\subsubsection*{Course Topics}
\begin{enumerate}
    \item \textbf{Introduction} (2 sessions)
    \item \textbf{Types of Languages and Grammars} (1 session)
    \item \textbf{Lexical Analysis and Error Handling} (3 sessions)
    \item \textbf{Top-Down Syntactic Analysis} (5 sessions)
    \begin{itemize}
        \item Recursive Descent Parsing
        \item LL(1) Parsing
        \item Error Handling in Syntax Analysis
    \end{itemize}
    \item \textbf{Bottom-Up Syntactic Analysis} (8 sessions)
    \begin{itemize}
        \item Operator Precedence
        \item Simple Precedence
        \item LR(1) Parsing including SLR(1), LALR(1), and CLR(1)
    \end{itemize}
    \item \textbf{Semantic Analysis} (1 session)
    \item \textbf{Symbol Table Management} (1 session)
    \item \textbf{Run-Time Memory Allocation Techniques} (2 sessions)
    \item \textbf{Code Generation} (5 sessions)
    \item \textbf{Code Optimization and Refinement} (1 session)
    \item \textbf{Automatic Compiler Generation} (1 session)
\end{enumerate}

\subsubsection*{Evaluation}
\begin{itemize}
    \item Midterm Exam: 35\%
    \item Final Exam: 35\%
    \item Practical Project: 20\%
    \item Quizzes and Exercises: 10\%
\end{itemize}

\subsubsection*{References}
\begin{enumerate}
    \item A. Aho, M. Lam, R. Sethi, and J. Ullman, \textit{Compilers: Principles, Techniques, and Tools}, 2nd Edition, Addison Wesley, 2007.
    \item D. Grune, H. Bal, C. Jacobs, and K. Langendoen, \textit{Modern Compiler Design}, John Wiley, 2001.
    \item J. Tremblay and P. Sorenson, \textit{Theory and Practice of Compiler Writing}, McGraw Hill, 1985.
    \item C. Fisher and R. LeBlanc, \textit{Crafting a Compiler with C}, Benjamin Cummings, 1991.
\end{enumerate}

\newpage

\section{Theory of Machines and Languages}
\subsection*{Course Code: 40415, Units: 3, Prerequisite: Data Structures and Algorithms}

\subsubsection*{Course Objectives}
This course covers theoretical aspects of computer engineering and computer science. The topics include various computational models, their computational power, computational properties, and their applications. Other topics include the concepts of computability, decidability, and the Church-Turing thesis regarding algorithms.

\subsubsection*{Course Topics}
\begin{enumerate}
    \item \textbf{Preliminary Topics} (4 sessions)
    \begin{itemize}
        \item Propositional Logic, Predicate Logic, Proof Systems, Set Theory, Russell's Paradox, Countable and Uncountable Sets, Languages, and Grammars.
    \end{itemize}
    
    \item \textbf{Finite State Machines} (8 sessions)
    \begin{itemize}
        \item Deterministic Finite Automata (DFA), Non-deterministic Finite Automata (NFA), Regular Languages, Regular Expressions, Right Linear Grammars, Left Linear Grammars, Regular Grammars, Context-free Grammars, Non-regular Languages, Pumping Lemma for Regular Languages.
    \end{itemize}

    \item \textbf{Context-Free Languages} (10 sessions)
    \begin{itemize}
        \item Context-Free Grammars, Context-Free Languages, Leftmost Derivations, Rightmost Derivations, Derivation Trees, Ambiguous Grammars, Unambiguous Grammars, Intrinsically Ambiguous Languages, Non-ambiguous Languages, Simplification of Context-Free Grammars, Chomsky Normal Form, Greibach Normal Form, Membership Problem, CYK Algorithm, Pushdown Automata, Equivalence of Pushdown Automata and Context-Free Grammars, Deterministic Pushdown Automata, Deterministic Context-Free Languages, Non-context-free Languages, Pumping Lemma for Context-Free Languages.
    \end{itemize}

    \item \textbf{Computability} (8 sessions)
    \begin{itemize}
        \item Turing Machines, Church-Turing Thesis, Decidability and Undecidability, Computability and Uncomputability, Halting Problem, Post Correspondence Problem, Computational Complexity, Class P, Class NP, NP-Complete Problems, NP-Hard Problems.
    \end{itemize}
\end{enumerate}

\subsubsection*{Evaluation}
\begin{itemize}
    \item Weekly Exercises: 30\%
    \item Quizzes: 45\%
    \item Final Exam: 25\%
\end{itemize}

\subsubsection*{References}
\begin{enumerate}
    \item M. Sipser, \textit{Introduction to the Theory of Computation}, 3rd Edition, Cengage Learning, 2013.
    \item P. Linz, \textit{An Introduction to Formal Languages and Automata}, 3rd Edition, Jones and Bartlett Publishers, 2001.
    \item J. E. Hopcroft, R. Motwani, and J. D. Ullman, \textit{Introduction to Automata Theory, Languages, and Computation}, 2nd Edition, Addison-Wesley, 2001.
    \item J. P. Denning, J. B. Dennis, and J. E. Qualitz, \textit{Machines, Languages, and Computation}, Prentice-Hall, 1978.
    \item J. E. Hopcroft and J. D. Ullman, \textit{Introduction to Automata Theory, Languages, and Computation}, Addison-Wesley, 1979.
    \item P. J. Cameron, \textit{Sets, Logics and Categories}, Springer, 1998.
\end{enumerate}

\newpage

\section{Artificial Intelligence}
\subsection*{Course Code: 40417, Units: 3, Prerequisite: Data Structures and Algorithms, Engineering Probability and Statistics}

\subsubsection*{Course Objectives}
This course introduces both theoretical and practical aspects of Artificial Intelligence (AI). The goal of the course is to introduce techniques for making optimal or near-optimal decisions in various problems and environments. The course covers topics such as search, problem-solving, knowledge representation, and inference. It also addresses search in uncertain environments, knowledge representation in these environments, and probabilistic inference for decision-making under uncertainty. Furthermore, an introduction to machine learning will be provided. The course will conclude with an introduction to several application areas of AI.

\subsubsection*{Course Topics}
\begin{enumerate}
    \item \textbf{Introduction to AI and its History}
    \item \textbf{Introduction to Intelligent Agents}
    \item \textbf{Uninformed Search}:
    \begin{itemize}
        \item Breadth-First Search (BFS), Depth-First Search (DFS)
        \item Iterative Deepening, Uniform Cost Search
    \end{itemize}
    \item \textbf{Informed Search}:
    \begin{itemize}
        \item Admissible and Consistent Heuristics
        \item Greedy Best-First Search
        \item A* Algorithm and Optimality Proof
        \item Heuristic Function Generation
    \end{itemize}
    \item \textbf{Local Search}:
    \begin{itemize}
        \item Hill-Climbing, Simulated Annealing, Local Beam Search, Genetic Algorithms
        \item Gradient Descent in Continuous Spaces
    \end{itemize}
    \item \textbf{Constraint Satisfaction Problems (CSP)}:
    \begin{itemize}
        \item Backtracking Search, Techniques like LCV, MRV, Forward Checking, MAC, AC3
        \item Local Search for CSP
    \end{itemize}
    \item \textbf{Adversarial Search}:
    \begin{itemize}
        \item Minimax Algorithm and Alpha-Beta Pruning
        \item Expectiminimax Algorithm
    \end{itemize}
    \item \textbf{Markov Decision Processes (MDP)}:
    \begin{itemize}
        \item Policy Evaluation and Improvement
        \item Value Iteration and Policy Iteration
    \end{itemize}
    \item \textbf{Reinforcement Learning}:
    \begin{itemize}
        \item Model-Based Methods, Temporal Difference Learning, Q-Learning Algorithm
    \end{itemize}
    \item \textbf{Logic}:
    \begin{itemize}
        \item Propositional Logic and Inference (including Resolution)
        \item First-Order Logic and Inference
    \end{itemize}
    \item \textbf{Bayesian Networks}:
    \begin{itemize}
        \item Representation in Bayesian Networks, Independence in Networks
        \item Exact and Approximate Inference using Sampling
        \item Parameter Estimation in Bayesian Networks
        \item Applications: Markov Models, Hidden Markov Models, Naïve Bayes Classifier
    \end{itemize}
    \item \textbf{Introduction to Machine Learning}
    \begin{itemize}
        \item Linear Models, Neural Networks
    \end{itemize}
    \item \textbf{Applications of AI}:
    \begin{itemize}
        \item Natural Language Processing
        \item Computer Vision
        \item Robotics
    \end{itemize}
\end{enumerate}

\subsubsection*{Evaluation}
\begin{itemize}
    \item Theoretical and Practical Exercises: 6 points
    \item Midterm Exam: 5 points
    \item Final Exam: 7 points
    \item Quizzes: 2 points
\end{itemize}

\subsubsection*{References}
\begin{itemize}
    \item Stuart Russell and Peter Norvig, \textit{Artificial Intelligence: A Modern Approach}, 3rd Edition, 2009.
\end{itemize}

\newpage

\section{Computer Networks Lab}
\subsection*{Course Code: 40416, Units: 1, Prerequisite: None, Corequisite: Computer Networks}

\subsubsection*{Course Objectives}
The Computer Networks Lab, offered for undergraduate students, serves as a complementary course to the Computer Networks lecture. In this course, students will gain practical experience with key concepts learned in the Computer Networks lecture. The lab is conducted in ten 3-hour sessions.

\subsubsection*{Course Topics}
\begin{enumerate}
    \item \textbf{Introduction to Basic Computer Network Concepts}
    \item \textbf{Review of Layered Architecture}
    \item \textbf{Physical Connection of Machines and Types of Transmission Cables}
    \item \textbf{Network Cable Socketing}
    \item \textbf{Introduction to Wireshark Software}
    \item \textbf{Examination of HTTP Communication}
    \item \textbf{Examination of TelNet Communication}
    \item \textbf{Analysis of DNS Requests and Responses}
    \item \textbf{Advanced Wireshark Usage}
    \item \textbf{Configuring and Setting Up a DNS Server}
    \item \textbf{Introduction to Routers and Switches}
    \item \textbf{Introduction to Packet Tracer Software}
    \item \textbf{Cisco Router and Switch Commands}
    \item \textbf{Introduction to GNS3 Software}
    \item \textbf{IP Addressing and IP Subnetting}
    \item \textbf{Implementation of a Static Routing Scenario in Packet Tracer}
    \item \textbf{Dynamic Routing}
    \item \textbf{Configuring RIP Routing Protocol in Packet Tracer}
    \item \textbf{Configuring OSPF Routing Protocol in Packet Tracer}
    \item \textbf{Introduction to NAT Mechanism}
    \item \textbf{Configuring Static NAT}
    \item \textbf{Configuring Dynamic NAT}
    \item \textbf{Configuring PAT}
    \item \textbf{Introduction to BGP Routing Protocol}
    \item \textbf{Implementing a BGP Protocol Scenario}
\end{enumerate}

\subsubsection*{Evaluation}
\begin{itemize}
    \item Laboratory Activities and Reports: 15 points
    \begin{itemize}
        \item 15 points will be awarded for completing the lab experiments during the sessions and submitting the corresponding reports before the next session.
        \item Each session's score will be divided equally between the lab activity and the submission of the report.
        \item Failure to attend a session results in losing the points for that session and report.
        \item Before each session, students must read the experiment instructions and, if necessary, review related content from the Computer Networks lecture.
    \end{itemize}
    \item Final Exam: 5 points
\end{itemize}

\subsubsection*{References}
\begin{itemize}
    \item James Kurose and Keith Ross, \textit{Computer Networking: A Top-Down Approach}, 7th Edition, Pearson, 2016.
    \item Larry L. Peterson and Bruce S. Davie, \textit{Computer Networks: A Systems Approach}, 5th Edition, 2011.
    \item Andrew Tanenbaum, \textit{Computer Networks}, 5th Edition, Pearson, 2010.
\end{itemize}

\newpage

\section{Operating Systems}
\subsection*{Course Code: 40424, Units: 3, Prerequisite: Computer Architecture}

\subsubsection*{Course Objectives}
The goal of this course is to familiarize undergraduate students with the principles of operating systems. The course includes four individual programming assignments that introduce students to system programming. Additionally, three group programming assignments will familiarize students with kernel-level programming.

\subsubsection*{Course Topics}
\begin{enumerate}
    \item \textbf{Introduction to Operating Systems} (2 sessions)
    \begin{itemize}
        \item Basic concepts of operating systems
        \item Structure and components of operating systems
        \item Process, address space, I/O, and dual-mode operations
        \item System architecture and structure
    \end{itemize}
    
    \item \textbf{Process Management} (3 sessions)
    \begin{itemize}
        \item Single-threaded, multi-threaded processes, forked processes, and process control blocks
        \item Interrupt management
        \item Process communication
    \end{itemize}

    \item \textbf{Concurrency and Synchronization} (3 sessions)
    \begin{itemize}
        \item Critical regions and mutual exclusion
        \item Indivisible operations
        \item Locks, semaphores, and monitors
    \end{itemize}
    
    \item \textbf{Scheduling} (3 sessions)
    \begin{itemize}
        \item Scheduling algorithm objectives
        \item First-Come-First-Served (FCFS), Round Robin (RR), Shortest Job Next (SJN), and Least Time Remaining First (LTRF)
        \item Real-time scheduling
    \end{itemize}

    \item \textbf{Deadlock and Starvation} (2 sessions)
    \begin{itemize}
        \item Conditions for deadlock occurrence
        \item Methods for deadlock handling, detection, and prevention
        \item Dining Philosophers problem and Banker's algorithm
    \end{itemize}

    \item \textbf{Memory Management} (2 sessions)
    \begin{itemize}
        \item Memory management, paging, segmentation, and combination of paging and segmentation
        \item Address translation, page table, two-level and multi-level paging, and inverted page tables
        \item Translation Lookaside Buffer (TLB)
    \end{itemize}

    \item \textbf{Virtual Memory} (2 sessions)
    \begin{itemize}
        \item Demand paging
        \item Page frame allocation and page faults
        \item Page replacement algorithms (FIFO, Least Recently Used, Random, Not Recently Used, Clock, and Nth Chance)
        \item Working set and thrashing
    \end{itemize}

    \item \textbf{Mass Storage Systems} (2 sessions)
    \begin{itemize}
        \item Types of I/O devices, controllers, and device drivers
        \item Storage devices (HDD and SSD)
        \item Disk scheduling algorithms (FCFS, Shortest Seek Time First, SCAN, and C-SCAN)
    \end{itemize}

    \item \textbf{File Systems} (3 sessions)
    \begin{itemize}
        \item Disk management methods and components of file systems
        \item File Allocation Table, UNIX file system, and NTFS
        \item Memory-mapped files and caching in file systems
    \end{itemize}

    \item \textbf{Protection and Security} (1 session)
    \item \textbf{Virtual Machines} (1 session)
\end{enumerate}

\subsubsection*{Evaluation}
\begin{itemize}
    \item Midterm and final exams: 40\% of the total grade
    \item Four individual programming assignments due throughout the semester: 25\% of the total grade
    \item Three group programming assignments due throughout the semester: 35\% of the total grade
\end{itemize}

\subsubsection*{References}
\begin{itemize}
    \item A. Silberschatz, P. B. Galvin, and G. Gagne, \textit{Operating System Concepts}, 10th Edition, Wiley Publishing, 2018.
    \item T. Anderson and M. Dahlin, \textit{Operating Systems: Principles and Practice}, 2nd Edition, Recursive Books, 2014.
\end{itemize}

\newpage

\section{Data and Network Security}
\subsection*{Course Code: 40442, Units: 3, Prerequisite: Computer Networks (40443)}

\subsubsection*{Course Objectives}
The objective of this course is to familiarize students with the fundamental concepts of security, defense mechanisms, and attacks in the areas of system, web, network, and mobile security.

\subsubsection*{Course Topics}
\begin{enumerate}
    \item \textbf{Basic Concepts and Definitions}
    \begin{itemize}
        \item Security policies and access control models
        \item Covert channels, information flow control
        \item Discretionary Access Control (DAC), Mandatory Access Control (MAC) models
        \item Role-Based Access Control (RBAC)
    \end{itemize}
    
    \item \textbf{System Security}
    \begin{itemize}
        \item Software execution methods and system interactions, vulnerabilities
        \item Attacks and defensive methods (control hijacking)
        \item Secure management of legacy code (sandboxing, virtualization, isolation at various layers)
        \item Secure code development methods (static analysis, dynamic analysis)
        \item Security violations and Fuzzing
    \end{itemize}

    \item \textbf{Web Security Model}
    \begin{itemize}
        \item Security of web application software (SQL injection, XSS, CSRF)
        \item Web session management (Cookies)
        \item Symmetric and asymmetric cryptography concepts
        \item Message integrity codes and hash functions
        \item Web information security during transfer (Https/SSL)
        \item Defensive mechanisms on the browser side (SOP, CSP, CORS)
    \end{itemize}
    
    \item \textbf{Network Security}
    \begin{itemize}
        \item Security threats in network protocols (routing, BGP, DNS, TCP, etc.)
        \item Defensive tools in networks (IDS, VPN, Firewalls)
        \item Denial of Service (DoS) attacks and defensive strategies
        \item Trusted Computing and SGX
    \end{itemize}

    \item \textbf{Mobile Security}
    \begin{itemize}
        \item Security of mobile platforms (iOS, Android)
        \item Mobile threats
    \end{itemize}
\end{enumerate}

\subsubsection*{Evaluation}
\begin{itemize}
    \item Theoretical exercises: 8\% of the total grade
    \item Midterm and final exams: 10\% of the total grade
    \item Quizzes: 2\% of the total grade
\end{itemize}

\subsubsection*{References}
\begin{itemize}
    \item Matt Bishop, \textit{Computer Security}, Addison-Wesley, 2017.
    \item John Erickson, \textit{The Art of Exploitation}, 2nd Edition, No Starch Press, 2008.
    \item Robert C. Seacord, \textit{Secure Coding in C and C++}, 2nd Edition, Pearson Education, 2005.
    \item A. Sotirov, \textit{Bypassing Browser Memory Protections}, 2008.
    \item T. Garfinkel, \textit{Traps and Pitfalls: Practical Problems in System Call Interposition Based Security Tools}, NDSS, 2003.
    \item Adam Barth, Collin Jackson, and John C. Mitchell, \textit{Securing Browser Frame Communication}, Usenix, 2008.
    \item Adam Barth, Collin Jackson, Charles Reis, and the Google Chrome Team, \textit{The Security Architecture of the Chromium Browser}, 2008.
    \item Bortz et al., \textit{Origin Cookies: Session Integrity for Web Applications}, 2011.
    \item Enck, Ongtang, and McDaniel, \textit{Understanding Android Security}, 2009.
    \item Allan Tomlinson, \textit{Introduction to the TPM: Smart Cards, Tokens, Security and Applications}, 2008.
    \item Andrew Baumann, Marcus Peinado, and Galen Hunt, \textit{Shielding Applications from an Untrusted Cloud with Haven}, OSDI 2014.
\end{itemize}

\newpage

\section{Computer Networks}
\subsection*{Course Code: 40443, Units: 3, Prerequisite: Engineering Statistics and Probability, Co-requisite: Operating Systems}

\subsubsection*{Course Objectives}
The objective of this course is to introduce students to the fundamental concepts of computer networks and related topics.

\subsubsection*{Course Topics}
\begin{enumerate}
    \item \textbf{Socket Programming}
    \item \textbf{IP Packet Switching}
    \item \textbf{IP Addressing and Routing}
    \item \textbf{Transmission Protocols (TCP and UDP)}
    \item \textbf{Congestion Control}
    \item \textbf{Address Translation (DNS, DHCP, and ARP)}
    \item \textbf{Middleware}
    \item \textbf{Switches and Bridges}
    \item \textbf{Links}
    \item \textbf{Connection-Oriented Routing}
    \item \textbf{Distance Vector Routing and Path Vector Routing}
    \item \textbf{Policy-Based Path Vector Routing (BGP)}
    \item \textbf{Subnet Networks and Peer-to-Peer Networks}
    \item \textbf{Multimedia Streaming}
    \item \textbf{Circuit Switching}
    \item \textbf{Wireless and Mobile Networks}
    \item \textbf{Content Delivery Networks (CDN)}
    \item \textbf{Software-Defined Networking (SDN)}
\end{enumerate}

\subsubsection*{Evaluation}
\begin{itemize}
    \item Theoretical exercises: 8\% of the total grade
    \item Midterm and final exams: 10\% of the total grade
    \item Quizzes: 2\% of the total grade
\end{itemize}

\subsubsection*{References}
\begin{itemize}
    \item Larry L. Peterson and Bruce S. Davie, \textit{Computer Networks: A Systems Approach}, 5th Edition, 2011.
\end{itemize}

\newpage

\section{Game Theory}
\subsection*{Course Code: 40456, Units: 3, Prerequisite: Data Structures and Algorithms, Engineering Statistics and Probability}

\subsubsection*{Course Objectives}
Game theory has wide applications in various fields, especially economics, business, and social sciences. Generally, in game theory, we deal with systems that involve intelligent and self-interested agents, each of whom changes the system's state according to their own interests. Game theory provides the tools to analyze such systems and helps us control them in a logical and structured way. This course aims to introduce students to the basic concepts of game theory and a few examples of its applications in modeling, mathematical analysis, and simulation.

\subsubsection*{Course Topics}
\begin{enumerate}
    \item \textbf{Normal Form Games} (4 sessions)
    \begin{itemize}
        \item Rational Behavior and Utility Function
        \item Definition of Normal Form Games
        \item Simple and Mixed Nash Equilibria
        \item Examples of Classic Normal Form Games
        \item Methods to Compute Equilibria in Simple Normal Form Games
    \end{itemize}
    \item \textbf{Extensive Form Games} (2 sessions)
    \begin{itemize}
        \item Definition of Extensive Form Games
        \item Subgame Perfect Equilibrium
        \item Examples of Classic Extensive Form Games
        \item Methods to Compute Equilibria in Simple Extensive Form Games
    \end{itemize}
    \item \textbf{Evolutionary Game Theory} (2 sessions)
    \begin{itemize}
        \item Evolutionarily Stable Strategies
        \item Relationship to Nash Equilibria
        \item Mixed Evolutionarily Stable Strategies
    \end{itemize}
    \item \textbf{Braess Paradox and Traffic Modeling using Game Theory} (1 session)
    \begin{itemize}
        \item Game Theory in Traffic Modeling
        \item Nash Equilibrium in Traffic Systems
        \item Braess Paradox
    \end{itemize}
    \item \textbf{Matching Markets} (2 sessions)
    \begin{itemize}
        \item Bipartite Graphs
        \item Perfect Matchings
        \item Market Clearing Prices
        \item Relationship to Auctions
    \end{itemize}
    \item \textbf{Bargaining Models} (2 sessions)
    \begin{itemize}
        \item Modeling Human Interaction (Nash Bargaining Solution, Final Game)
        \item Modeling Exchange between Two Humans (Stable Outcomes, Equilibrium Outcomes)
    \end{itemize}
    \item \textbf{Auction Mechanism Design} (5 sessions)
    \begin{itemize}
        \item Definition of Auctions and Game-Theoretic Modeling
        \item Types of Auctions and Their Relations (German, Japanese, English, First Price, Second Price)
        \item Analysis of Second Price Auctions
        \item Introduction to VCG and Sponsored Search Auctions
    \end{itemize}
    \item \textbf{Simple Networked Trade Models with Intermediaries} (2 sessions)
    \begin{itemize}
        \item Pricing in Markets
        \item Game-Theoretic Modeling of Trade on Intermediated Networks
        \item Equilibrium Points and Relationship with Auctions
    \end{itemize}
    \item \textbf{Signaling Games and Information Cascades} (3 sessions)
    \begin{itemize}
        \item Signaling Games
        \item Speech-Act Theory
        \item Bayes' Theorem and Decision Making under Uncertainty
        \item Herding Behavior
        \item Information Cascades
    \end{itemize}
    \item \textbf{Market Analysis, Network Effects, and Externalities} (2 sessions)
    \begin{itemize}
        \item Market Analysis without Network Effects
        \item Market Analysis with Network Effects
        \item Dynamic View of Markets and Stable vs. Unstable Points
        \item Positive and Negative Externalities
    \end{itemize}
    \item \textbf{Social Choice and Voting Mechanisms} (2 sessions)
    \begin{itemize}
        \item Definition of Social Choice and Voting Mechanisms
        \item Familiar Voting Mechanisms
        \item Arrow’s Impossibility Theorem
    \end{itemize}
    \item \textbf{Asset Valuation and Intellectual Property} (2 sessions)
    \begin{itemize}
        \item Externalities and Coase Theorem
        \item Tragedy of the Commons
        \item Intellectual Property
    \end{itemize}
    \item \textbf{Introduction to Coalitional Game Theory} (1 session)
    \begin{itemize}
        \item Definition of Coalitional Games
        \item Core Concept
        \item Solving Classic Coalitional Games
        \item Shapley Value
    \end{itemize}
\end{enumerate}

\subsubsection*{Evaluation}
\begin{itemize}
    \item Theoretical Exercises: 20\% of the total grade
    \item Exams (Midterm, Final, and Quizzes): 80\% of the total grade
\end{itemize}

\subsubsection*{References}
\begin{itemize}
    \item Yoav Shoham and Kevin Leyton-Brown, \textit{Multiagent Systems: Algorithmic, Game-Theoretic, and Logical Foundations}, Cambridge University Press, 2008.
    \item David Easley and Jon Kleinberg, \textit{Networks, Crowds, and Markets: Reasoning about a Highly Connected World}, Cambridge University Press, 2010.
    \item Martin J. Osborne and Ariel Rubinstein, \textit{A Course in Game Theory}, MIT Press, 1994.
\end{itemize}

\newpage

\section{Software Engineering}
\subsection*{Course Code: 40474, Units: 3, Prerequisite: Systems Analysis and Design}

\subsubsection*{Course Objectives}
This course focuses on the engineering principles that must be adhered to throughout the software development process. Students have already been introduced to software construction (programming), requirements analysis, and software design in previous courses. The goal of this course is not to teach new methods for requirements analysis or software design but to emphasize the production of software as an engineering product, similar to other products produced in engineering fields. Initially, the difference between products produced through engineering methods and those produced through artistic methods will be discussed. Then, the expectations that an engineering product must meet will be clarified. The course further emphasizes engineering production methods such as modeling, measurability, evaluation, verification, and validation of intermediate products. Since students have been less familiar with formal descriptions of requirements, measurement, estimation, and testing in previous courses, these topics will receive additional focus in this course. Finally, activities such as project management, scheduling, risk management, configuration management, and quality assurance will be reviewed, emphasizing their role in producing software as an engineering product.

\subsubsection*{Course Topics}
\begin{enumerate}
    \item \textbf{Introduction} (2 sessions)
    \item \textbf{Process Models} (2 sessions)
    \item \textbf{Agile Development} (1 session)
    \item \textbf{Understanding Requirements} (1 session)
    \item \textbf{Formal Methods} (5 sessions)
    \item \textbf{Design Concepts} (1 session)
    \item \textbf{Architectural Design} (1 session)
    \item \textbf{Interface Design} (1 session)
    \item \textbf{Pattern-Based Design} (1 session)
    \item \textbf{Testing Strategies} (1 session)
    \item \textbf{Testing Methods} (4 sessions)
    \item \textbf{Product Measurement} (1 session)
    \item \textbf{Process and Project Measurement} (1 session)
    \item \textbf{Estimation} (1 session)
    \item \textbf{Quality Concepts} (1 session)
    \item \textbf{Review Methods} (1 session)
    \item \textbf{Quality Assurance} (1 session)
    \item \textbf{Configuration Management} (1 session)
    \item \textbf{Project Management} (1 session)
    \item \textbf{Scheduling} (1 session)
    \item \textbf{Risk Management} (1 session)
\end{enumerate}

\subsubsection*{Evaluation}
\begin{itemize}
    \item 3 Practical-Theoretical Exercises during the term: 20\% of the total grade
    \item 3 Multiple Choice Tests on the course materials during the term: 30\% of the total grade
    \item Approximately 5 Short Quizzes during the term: 10\% of the total grade (bonus points)
    \item Final Exam (Descriptive and Multiple Choice): 50\% of the total grade
    \item Optional Seminar on topics not covered in class but related to the course content (with prior approval): 10\% of the total grade (bonus points)
\end{itemize}

\subsubsection*{References}
\begin{itemize}
    \item R. S. Pressman, \textit{Software Engineering: A Practitioner’s Approach}, 8th Edition, McGraw-Hill, 2014.
    \item P. Ammann and J. Offutt, \textit{Introduction to Software Testing}, Cambridge University Press, 2008.
    \item J. Woodcock and J. Davies, \textit{Using Z: Specification, Refinement, and Proof}, Prentice-Hall, 1996.
\end{itemize}

\newpage

\section{Computer Simulation}
\subsection*{Course Code: 40634, Units: 3, Prerequisite: Engineering Probability and Statistics}

\subsubsection*{Course Objectives}
The goal of this course is to familiarize students with various simulation methods and related topics.

\subsubsection*{Course Topics}
\begin{enumerate}
    \item \textbf{Introduction to Simulation}
    \item \textbf{Familiarization with MATLAB or Similar Tools as Computational Tools for the Course}
    \item \textbf{Basic Principles and Examples of Simulation}
    \item \textbf{Discrete Event System Simulation Concepts}
    \item \textbf{Several Examples of Simulation}
    \item \textbf{Discrete-Event Simulation System Implementation Patterns}
    \item \textbf{Types of Discrete Event Simulation System Structures}
    \item \textbf{Sorted List Processing}
    \item \textbf{Methods of Drawing Systems for Simulation}
    \item \textbf{Statistical Models in Simulation}
    \item \textbf{Brief Review of Statistics and Probability}
    \item \textbf{Discrete Distributions}
    \item \textbf{Continuous Distributions}
    \item \textbf{Empirical Distributions}
    \item \textbf{Uniform Random Number Generation}
    \item \textbf{Required Characteristics for Random Numbers}
    \item \textbf{Methods of Random Number Generation}
    \item \textbf{Randomness Tests for Sequences}
    \item \textbf{Generation of Random Variables}
    \item \textbf{Inverse Transformation Method}
    \item \textbf{Acceptance-Rejection Method}
    \item \textbf{Combination}
    \item \textbf{Convolution}
    \item \textbf{Arrival Modeling}
    \item \textbf{Data Collection}
    \item \textbf{Evaluation of Sample Independence}
    \item \textbf{Distribution Fitting from Data}
    \item \textbf{Parameter Estimation}
    \item \textbf{Goodness-of-Fit Testing}
    \item \textbf{Model Selection in the Absence of Data Samples}
    \item \textbf{Input Process Models}
    \item \textbf{Verification and Validation of Simulation Models}
    \item \textbf{Analysis of Output Data}
    \item \textbf{Transient and Steady-State Behavior of Stochastic Processes}
    \item \textbf{Types of Simulation Based on Output Analysis}
    \item \textbf{Statistical Analysis of Steady-State Parameters}
    \item \textbf{Design of Experiments and Sensitivity Analysis}
    \item \textbf{Advanced Topics in Simulation}
    \item \textbf{Monte Carlo Simulation}
    \item \textbf{Real-World Examples of Simulation}
\end{enumerate}

\subsubsection*{Evaluation}
\begin{itemize}
    \item Theoretical Exercises: 3 points
    \item Midterm and Final Exams: 15 points
    \item Quizzes: 2 points
\end{itemize}

\subsubsection*{References}
\begin{itemize}
    \item Banks, Carson, Nelson, and Nicol. \textit{Discrete-Event System Simulation}, 5th Edition, Prentice-Hall, 2010.
\end{itemize}

\newpage

\section{Computer Engineering Project}
\subsection*{Course Code: 40760, Units: 3, Prerequisite: Presentation of Scientific and Technical Topics}

\subsubsection*{Course Objectives}
The goal of the undergraduate project is to analyze, design, and implement a real project or conduct a research project based on the concepts learned throughout the undergraduate program.

\newpage
\section{Signals and Systems}
\subsection*{Course Code: 40242, Units: 3, Prerequisite: Foundations of Electrical and Electronic Circuits, Co-requisite: None}

\textbf{Course Summary:} 
\begin{quote}
   This course aims to familiarize students with modeling, describing, and analyzing signals and systems in both time and frequency domains from theoretical and practical perspectives. In addition to theoretical exercises, MATLAB practices are included to reinforce practical understanding of the presented concepts.
\end{quote}

\textbf{Course Outline:}
\begin{itemize}
    \item Introduction
    \item Continuous-time and Discrete-time Signals
    \item Signal Transformations
    \item Signal Properties and Types (periodic, even, odd, etc.)
    \item System Properties (memoryless, causal, stable, linear, time-invariant)
    \item Linear Time-Invariant (LTI) Systems
    \item Convolution (Continuous and Discrete)
    \item Impulse Response
    \item Linear Constant-Coefficient Differential Equations (LCCDE) and Block Diagrams
    \item Fourier Series for Periodic Signals
    \item LTI System Response to Complex Exponentials
    \item Fourier Series Representations for Continuous and Discrete-Time Periodic Signals
    \item Fourier Series Properties (linearity, time shift, time scaling, etc.)
    \item Fourier Series and LTI Systems: System Function and Frequency Response
    \item Continuous-Time Fourier Transform (CTFT)
    \item Fourier Transform for Aperiodic and Periodic Signals
    \item CTFT Properties (linearity, time shift, etc.)
    \item Multiplication and Convolution
    \item Systems Described by LCCDE
    \item Discrete-Time Fourier Transform (DTFT)
    \item DTFT for Aperiodic and Periodic Signals
    \item DTFT Properties (periodicity, linearity, time shift, etc.)
    \item Multiplication and Convolution
    \item Systems Described by LCCDE
    \item Time/Frequency Description of Signals and Systems
    \item Magnitude and Phase of the Fourier Transform
    \item Frequency Response Magnitude and Phase
    \item Log Magnitude Plots
    \item Bode Plots
    \item Ideal and Non-Ideal Filters
    \item First and Second Order Continuous and Discrete-Time Systems
    \item Sampling
    \begin{itemize}
        \item Sampling Theorem, Impulse Train, Interpolation, Aliasing
    \end{itemize}
    \item Laplace Transform
    \begin{itemize}
        \item Region of Convergence, Inverse Transform, Pole-Zero Diagram
        \item Properties (linearity, time shift, etc.)
        \item Time and s-domain Differentiation and Integration
        \item Initial and Final Value Theorems
        \item Causality and Stability
        \item Systems Described by LCCDE
        \item Butterworth Filter, Block Diagram Representation
        \item One-Sided Laplace Transform
    \end{itemize}
    \item z-Transform
    \begin{itemize}
        \item Region of Convergence, Inverse Transform, Pole-Zero Diagram
        \item Properties (linearity, time shift, etc.)
        \item Initial Value Theorem
        \item Causality and Stability
        \item Systems Described by LCCDE
        \item Block Diagram Representation
        \item One-Sided z-Transform
    \end{itemize}
\end{itemize}

\textbf{Books Used:}
\begin{itemize}
    \item Alan V. Oppenheim, Alan S. Willsky, and S. Hamid Nawab, \textit{Signals and Systems}, 2nd Edition, Prentice Hall, 1996.
\end{itemize}

\textbf{Evaluation:}
\begin{itemize}
    \item Exercises: 15\%
    \item Mid-term Exam: 35\%
    \item Final Exam: 50\%
\end{itemize}

\newpage

\section{Modern Information Retrieval}
\subsection*{Course Code: 40324, Units: 3, Prerequisite: Data Structures and Algorithms, Co-requisite: None}

\textbf{Course Summary:} 
\begin{quote}
    This course introduces information retrieval systems. It starts with indexing operations and the Boolean retrieval model. Then the vector space model and tf-idf representation are discussed, followed by techniques for speeding up document scoring and ranking. Probabilistic retrieval models are introduced along with concepts such as document classification, clustering, and learning to rank. The course continues with an introduction to web search engines and their key components such as crawlers, link graph analysis, and near-duplicate detection. Finally, recommender systems and advanced topics in information retrieval are covered.
\end{quote}

\textbf{Course Outline:}
\begin{itemize}
    \item Introduction to Information Retrieval
    \item Boolean Information Retrieval Systems and Indexing
    \item Document Preprocessing: Text Operations and Word Normalization
    \item Tolerant Retrieval
    \begin{itemize}
        \item Wild-card Queries, Spelling Correction
    \end{itemize}
    \item Blocked and Distributed Indexing
    \item Map-Reduce
    \item Index Compression
    \begin{itemize}
        \item Dictionary Compression
        \item Posting List Compression: Byte-wise and Gamma Encoding
    \end{itemize}
    \item Vector Space Model and tf-idf Representation
    \item Document Scoring and Ranking (Efficiency Improvements)
    \item Evaluation of Information Retrieval Systems and Metrics
    \item Probabilistic Information Retrieval Models
    \item Language Models
    \item Document Classification
    \begin{itemize}
        \item Naïve Bayes Classifier, Linear Classifiers
    \end{itemize}
    \item Document Clustering
    \begin{itemize}
        \item k-means, Hierarchical Clustering
    \end{itemize}
    \item Learning to Rank
    \item Dimensionality Reduction and Word Embeddings
    \begin{itemize}
        \item Latent Semantic Indexing (LSI), Word2Vec
    \end{itemize}
    \item Web Search Engines
    \begin{itemize}
        \item Crawlers, Near-Duplicate Detection
        \item Link Graph Analysis and PageRank
    \end{itemize}
    \item Recommender Systems
    \begin{itemize}
        \item Content-Based Methods
        \item Collaborative Filtering
        \item Hybrid Approaches
    \end{itemize}
    \item Advanced Topics
    \begin{itemize}
        \item Personalization, Information Retrieval in Social Networks
        \item Question Answering Systems
        \item Sentiment Analysis
        \item Cross-Lingual Information Retrieval
    \end{itemize}
\end{itemize}

\textbf{Books Used:}
\begin{itemize}
    \item C. D. Manning, P. Raghavan, and H. Schütze, \textit{Introduction to Information Retrieval}, Cambridge University Press, 2008.
\end{itemize}

\textbf{Evaluation:}
\begin{itemize}
    \item Mid-term Exam: 25\%
    \item Final Exam: 35\%
    \item Project: 25\%
    \item Short Exams: 10\%
    \item Quizzes: 5\%
\end{itemize}

\newpage

\section{Multimedia Systems}
\subsection*{Course Code: 40342, Units: 3, Prerequisite: Signals and Systems, Co-requisite: None}

\textbf{Course Summary:} 
\begin{quote}
   This course introduces students to the fundamental concepts of multimedia and multimedia systems, considering emerging value-added services. Topics include signal processing, compression, multimedia data types, system architectures, and network protocols for multimedia applications.
\end{quote}

\textbf{Course Outline:}
\begin{itemize}
    \item Introduction to Multimedia (2 sessions)
    \begin{itemize}
        \item Multimedia and Multimedia Systems
        \item Hypermedia
        \item Characteristics, Challenges, and Components of Multimedia Systems
        \item Multimedia Data, Projects, and Research Topics
    \end{itemize}
    
    \item Review of Signals and Systems (4 sessions)
    \begin{itemize}
        \item Discrete-Time Signals and Systems
        \item Sampling Theory
        \item Scalar and Vector Quantization
        \item Transform Domain Analysis
        \item FFT, STFT, and Wavelet Transforms
    \end{itemize}
    
    \item Audio (3 sessions)
    \begin{itemize}
        \item Audio Representation and Playback
        \item Sampling and Quantization
        \item Standards and Formats
        \item Temporal and Frequency Masking
        \item Audio Signal Processing
        \item Audio Compression
    \end{itemize}
    
    \item Entropy Coding (3 sessions)
    \begin{itemize}
        \item Lossy and Lossless Compression
        \item Run-Length Encoding
        \item Fixed-Length and Variable-Length Coding
        \item Huffman Coding
        \item Lempel-Ziv-Welch Coding
        \item Arithmetic Coding
    \end{itemize}
    
    \item Image (4 sessions)
    \begin{itemize}
        \item Color Spaces: YUV, RGB, HSV, CMYK
        \item Image Acquisition and Display
        \item Image Enhancement
        \item Image Compression: DCT, MPEG
    \end{itemize}
    
    \item Video (4 sessions)
    \begin{itemize}
        \item Basics of Analog and Digital Video
        \item Video Compression
        \item Intra-frame and Inter-frame Coding
        \item Motion Estimation and Compensation
        \item Video Quality Evaluation
        \item Video Coding Standards: MPEG1, MPEG2, MPEG4, H.261, H.263, H.264
    \end{itemize}
    
    \item Multimedia Systems (4 sessions)
    \begin{itemize}
        \item Standalone vs. Networked Systems
        \item Orchestrated vs. Live Systems
        \item System Components
        \item Real-Time Multimedia System Architecture
    \end{itemize}
    
    \item Multimedia Networking (3 sessions)
    \begin{itemize}
        \item Multimedia Data Delivery Quality
        \item Streaming Protocols
        \item Error Concealment
        \item Priority Encoding
        \item Overlay Networks
        \item Packet Loss, Congestion, and QoS
        \item Unicasting and Multicasting
        \item Wireless Multimedia
    \end{itemize}
    
    \item Multimedia Applications (3 sessions)
    \begin{itemize}
        \item Internet Telephony
        \item Digital Video Broadcasting
        \item IPTV and Interactive Television
        \item E-learning
        \item Human-Computer Interaction
        \item Multimedia Home Platforms
        \item Multimedia Information Retrieval Systems
        \item 3D Technologies
    \end{itemize}
\end{itemize}

\textbf{Books Used:}
\begin{itemize}
    \item R. Steinmetz and K. Nahrstedt, \textit{Multimedia: Computing, Communications and Applications}, Prentice Hall, 1995.
    \item R. Steinmetz and K. Nahrstedt, \textit{Multimedia Fundamentals: Media Coding and Content Processing}, Prentice Hall, 2002.
    \item K. R. Rao, Z. S. Bojkovic, and D. A. Milanovic, \textit{Multimedia Communication Systems: Techniques, Standards and Networks}, Prentice Hall, 2002.
\end{itemize}

\textbf{Evaluation:}
\begin{itemize}
    \item Theoretical Assignments: 3 points
    \item Mid-term and Final Exams: 12 points
    \item Quizzes: 3 points
\end{itemize}

\newpage

\section{Data Transmission}
\subsection*{Course Code: 40343, Units: 3, Prerequisite: Signals and Systems, Co-requisite: None}

\textbf{Course Summary:} 
\begin{quote}
   The goal of this course is to familiarize students with how data is transmitted through various media and by different methods, as well as the challenges and issues associated with each.
\end{quote}

\textbf{Course Outline:}
\begin{itemize}
    \item Transmission Media (6 sessions)
    \begin{itemize}
        \item Twisted Pair, Shielded Twisted Pair
        \item Coaxial Cable
        \item Waveguide
        \item Optical Fiber
        \item Free-Space Optical Link
        \item Microwave Link
        \item Satellite
    \end{itemize}
    
    \item Sources of Errors (4 sessions)
    \begin{itemize}
        \item Thermal Noise
        \item Electrical Noise (EMI, RFI)
        \item Attenuation Distortion
        \item Delay Distortion
        \item Echo
        \item Harmonic Distortion
        \item Intermodulation Distortion
        \item Crosstalk
        \item Fading
    \end{itemize}
    
    \item Error Detection and Correction (3 sessions)
    \begin{itemize}
        \item Longitudinal Redundancy Check (LRC)
        \item Vertical Redundancy Check (VRC)
        \item Two-Dimensional Parity Check (VRC-LRC)
        \item Cyclic Redundancy Check (CRC)
        \item Checksum
        \item Hamming Code
    \end{itemize}
    
    \item Types of Modulation (4 sessions)
    \begin{itemize}
        \item Analog Modulation
        \item Digital Modulation
        \item Pulse Modulation
    \end{itemize}
    
    \item Multiplexing (2 sessions)
    \begin{itemize}
        \item Time Division Multiplexing (TDM)
        \item Frequency Division Multiplexing (FDM)
        \item Code Division Multiplexing (CDM)
    \end{itemize}
    
    \item Multiple Access (2 sessions)
    \begin{itemize}
        \item Time Division Multiple Access (TDMA)
        \item Frequency Division Multiple Access (FDMA)
        \item Code Division Multiple Access (CDMA)
    \end{itemize}
    
    \item Channel Capacity (2 sessions)
    \begin{itemize}
        \item Shannon's Theorem
        \item Optimal Power Allocation
    \end{itemize}
    
    \item Data Compression (3 sessions)
    \begin{itemize}
        \item Audio Compression
        \item Huffman Coding
        \item Compression in Fax Systems
    \end{itemize}
    
    \item Switching (1 session)
    \begin{itemize}
        \item Circuit Switching
        \item Message Switching
        \item Packet Switching
    \end{itemize}
    
    \item Flow Control Efficiency (3 sessions)
    \begin{itemize}
        \item Stop-and-Wait Method
        \item Sliding Window Protocol
        \item Error Impact on Efficiency
    \end{itemize}
\end{itemize}

\textbf{Books Used:}
\begin{itemize}
    \item W. Stallings, \textit{Data and Computer Communications}, Prentice-Hall, 1996.
    \item F. Halsall, \textit{Data Communications, Computer Networks, and Open Systems}, 4th Edition, Addison Wesley, 1996.
    \item A. S. Tanenbaum, \textit{Computer Networks}, 3rd Edition, Prentice-Hall, 1996.
    \item Edham Sadeghi (Translator), \textit{Principles of Data Communication}, Tizhooshan-e-Sarzamin-e-Kohan Publishing, 2005. (in Persian)
\end{itemize}

\textbf{Evaluation:}
\begin{itemize}
    \item Theoretical Assignments: 4 points
    \item Mid-term and Final Exams: 16 points
\end{itemize}

\newpage

\section{Fundamentals of 3D Computer Vision}
\subsection*{Course Code: 40344, Units: 3, Prerequisite: Linear Algebra or Engineering Mathematics, Co-requisite: None}

\textbf{Course Summary:} 
\begin{quote}
   This course introduces students to fundamental concepts and methods for analyzing images to achieve a high-level understanding of their content. Topics include image formation and color representation, basic signal and image processing, 3D geometry, feature extraction, robust model fitting, clustering and segmentation, object recognition, nearest neighbor search, and deep learning techniques in computer vision.
\end{quote}

\textbf{Course Outline:}
\begin{itemize}
    \item Signal and Image Processing
    \begin{itemize}
        \item Basic Signal Processing Concepts
        \item Overview of Signals and Systems
        \item Convolution Function
        \item Fourier Transform
        \item Image Filtering
    \end{itemize}
    
    \item Basics of 3D Geometry
    \begin{itemize}
        \item Basic Geometric Concepts
        \item Brief Review of Linear Algebra
        \item Parametrization of Rotation Matrices
        \item Homogeneous Coordinates
        \item Pinhole Camera Model
        \item Mapping from Meters to Pixel Coordinates
    \end{itemize}
    
    \item Cameras and Projections
    \begin{itemize}
        \item Parallel and Perspective Projection
        \item Single-Axis Camera Rotations
        \item Simple Mosaic Image Construction
        \item Intrinsic and Extrinsic Camera Parameters
        \item General Camera Motion and Linear Mapping Estimation
        \item Camera Translation
    \end{itemize}
    
    \item 3D Reconstruction from Stereo Vision
    \begin{itemize}
        \item Surface Reconstruction and Rendering
        \item Point Cloud Triangulation
        \item Surface Mapping, Image-Based Rendering
        \item Planar Surfaces and Linear Homographies
    \end{itemize}
    
    \item Multi-Camera Systems
    \begin{itemize}
        \item Perspective Cameras and Bundle Adjustment
        \item Parallel Projection Cameras
        \item Camera Calibration and 3D Coordinate Systems
        \item Affine Structure
    \end{itemize}
    
    \item Keypoint Extraction
    \item Robust Model Fitting
    \item Clustering and Segmentation
    \begin{itemize}
        \item Graph Cuts
    \end{itemize}
    
    \item Object Recognition
    \begin{itemize}
        \item Template Classification
        \item Nearest Neighbors, PCA, Dimensionality Reduction
        \item Naive Bayes Classifier
        \item Ensemble of Simple Classifiers
        \item Neural Networks
    \end{itemize}
    
    \item Nearest Neighbors
    \item Deep Learning in Computer Vision
    \begin{itemize}
        \item Neural Networks and Backpropagation
        \item CNN Architectures: New Ideas, Advantages, and Limitations
        \item Spatio-temporal Deep Networks
        \item Training Deep Neural Networks using PyTorch
    \end{itemize}
\end{itemize}

\textbf{Books Used:}
\begin{itemize}
    \item Stefan Carlsson, \textit{Geometric Computing in Image Analysis and Visualization}, Lecture Notes, KTH University, 2007.
    \item Richard Szeliski, \textit{Computer Vision: Algorithms and Applications}, 1st Edition, Springer, 2010.
\end{itemize}

\textbf{Evaluation:}
\begin{itemize}
    \item Final Exam: 40\%
    \item Mid-term Exam: 15\%
    \item Quizzes: 10\%
    \item Assignments: 15\%
    \item Final Project: 20\%
\end{itemize}

\newpage

\section{Information Technology Ethics}
\subsection*{Course Code: 40347, Units: 3, Prerequisite: None, Co-requisite: None}

\textbf{Course Summary:} 
\begin{quote}
This course explores the ethical, legal, and social implications of information technology. It emphasizes the importance of professional conduct and ethical behavior in the development, deployment, and use of IT systems. Topics include ethical and legal frameworks, professional codes of conduct, privacy, security, intellectual property, digital rights, cybercrimes, social responsibility, and the philosophical underpinnings of ethics in the digital world.
\end{quote}

\textbf{Course Outline:}
\begin{itemize}
    \item Introduction (2 sessions)
    \begin{itemize}
        \item Course goals, structure, and teaching model
        \item Principles of ethics in engineering and IT
    \end{itemize}

    \item Relationship between ethics, etiquette, and law (1 session)
    \begin{itemize}
        \item Comparison between ethics and etiquette
        \item Compatibility and distinction with legal norms
        \item Law as retrospective, etiquette as forward-looking
    \end{itemize}

    \item History and types of ethical systems (5 sessions)
    \begin{itemize}
        \item Western ethical philosophies
        \item Islamic and Iranian ethics
        \item Global ethics, golden rules, and customary norms
    \end{itemize}

    \item Ethical dilemmas and resolution frameworks (3 sessions)
    \begin{itemize}
        \item Ethical conflicts and ambiguities
        \item Ethical judgment models and decision processes
    \end{itemize}

    \item Nature and application of professional codes (2 sessions)
    \begin{itemize}
        \item Charters, bylaws, and conduct codes
        \item Structure and evaluability of ethical documents
    \end{itemize}

    \item Models for code generation and refinement (2 sessions)
    \begin{itemize}
        \item Organizational and group-level codes
        \item Success metrics and code improvement models
    \end{itemize}

    \item Professional ethics and engineering systems (2 sessions)
    \begin{itemize}
        \item Professional engineering ethics in IT
        \item National and global engineering systems
    \end{itemize}

    \item Intellectual property in IT (3 sessions)
    \begin{itemize}
        \item Financial and creative rights
        \item Copyright, patents, and legal protections
        \item Future of digital rights and ethical frameworks
    \end{itemize}

    \item Ethics in the information and virtual society (3 sessions)
    \begin{itemize}
        \item Free information flow, citizen rights, and transparency
        \item Privacy, virtual reality, and social media ethics
        \item Democracy, digital health literacy, and public awareness
    \end{itemize}

    \item Cybercrimes and IT-related offenses (4 sessions)
    \begin{itemize}
        \item Malware, intrusion methods, and cyber threats
        \item Internet police, cybercrime laws, and social security
        \item Organizational security, open-source licensing, and privacy concerns
        \item Green IT and sustainable practices
    \end{itemize}

    \item Emerging issues in IT ethics (2 sessions)
    \begin{itemize}
        \item Network neutrality, augmented/virtual reality, and infosphere philosophy
        \item NBIC ethics and the Fourth Revolution (Floridi)
    \end{itemize}
\end{itemize}

\textbf{Books Used:}
\begin{itemize}
    \item George Reynolds, \textit{Ethics in Information Technology}, Thomson, 2011.
    \item Luciano Floridi, \textit{The Fourth Revolution: How the Infosphere is Reshaping Human Reality}, Oxford University Press, 2014.
    \item Luciano Floridi, \textit{Information and Computer Ethics}, Cambridge University Press, 2010.
    \item Ibo van de Poel, \textit{Ethics, Technology, and Engineering}, Wiley-Blackwell, 2011.
    \item Harris, M. J. Rabins, and C. E. Harris, \textit{Engineering Ethics: Concepts \& Cases}, Thomson, 2004.
    \item M. W. Martin, \textit{Ethics in Engineering}, McGraw-Hill, 2005.
    \item Duncan Langford, \textit{Internet Ethics}, Macmillan Press Ltd, 2000.
\end{itemize}

\textbf{Evaluation:}
\begin{itemize}
    \item Exercises (Comprehension and Problem Solving): 6 points
    \item Midterm and Final Exams: 12 points
    \item Quizzes: 2 points
\end{itemize}

\newpage

\section{VLSI Design}
\subsection*{Course Code: 40353, Units: 3, Prerequisite: Digital System Design, Basic Electrical and Electronic Circuits, Co-requisite: None}

\textbf{Course Summary:} 
\begin{quote}
This course introduces the design, analysis, and fabrication of VLSI (Very Large-Scale Integration) systems with a focus on transistor-level design. It covers the principles of modern VLSI circuit design, including MOS transistor modeling, logic gate implementation, power and delay analysis, and layout planning.
\end{quote}

\textbf{Course Outline:}
\begin{itemize}
    \item Introduction to VLSI Circuits
    \item VLSI Circuit Benchmarks and Abstraction Levels
    \item Chip Fabrication Process and Photolithography
    \item Layout Process and Design Rules
    \item Fabrication Defects and Manufacturing Issues
    \item Stick Diagrams
    \item MOS Transistor Characteristics
    \begin{itemize}
        \item I-V Characteristics of nMOS and pMOS
        \item DC Response, Body Effect, and Channel Length Modulation
        \item Leakage and Subthreshold Currents
        \item Latch-up Effect and Parasitics
    \end{itemize}
    \item On-chip Interconnects
    \begin{itemize}
        \item Wire Capacitance and Resistance
        \item Routing and Vias
    \end{itemize}
    \item Logic Gate and Combinational Circuit Design
    \begin{itemize}
        \item CMOS Logic, Pseudo-nMOS, Domino Logic
        \item Switch Logic, DCVS Logic
    \end{itemize}
    \item Power Consumption Analysis
    \begin{itemize}
        \item Static and Dynamic Power
    \end{itemize}
    \item Delay Calculation
    \begin{itemize}
        \item Logical Effort, Branch Effort, Path Delay
    \end{itemize}
    \item Sequential Elements
    \begin{itemize}
        \item Static and Dynamic Latches
        \item Clocking Structures
    \end{itemize}
    \item Arithmetic and Logic Elements
    \begin{itemize}
        \item Adders, Multipliers, Shifters, ALU
    \end{itemize}
    \item Floorplanning
    \item VLSI Circuit Testability
\end{itemize}

\textbf{Books Used:}
\begin{itemize}
    \item Wayne Wolf, \textit{Modern VLSI Design: System-on-Chip Design}, 3rd Edition, 2004.
    \item J. M. Rabaey, A. Chandrakasan, and B. Nikolić, \textit{Digital Integrated Circuits: A Design Perspective}, 2005.
    \item N. H. E. Weste and D. Harris, \textit{CMOS VLSI Design: A Circuits and Systems Perspective}, 3rd Edition, Addison-Wesley, 2005.
\end{itemize}

\textbf{Evaluation:}
\begin{itemize}
    \item Theoretical Exercises: 3 points
    \item Midterm and Final Exams: 15 points
    \item Quizzes: 2 points
\end{itemize}

\newpage

\section{Design of Algorithms}
\subsection*{Course Code: 40354, Units: 3, Prerequisite: Data Structures and Algorithms, Co-requisite: None}

\textbf{Course Summary:}
\begin{quote}
This course introduces common strategies for designing efficient algorithms for a variety of computational problems. Emphasis is placed on analyzing algorithm efficiency and proving correctness. Topics include classic algorithmic paradigms, complexity theory, graph algorithms, network flows, and approximation techniques.
\end{quote}

\textbf{Course Outline:}
\begin{itemize}
    \item Introduction and Sample Problems
    \begin{itemize}
        \item Solvability, Algorithm Analysis, Running Times
        \item Longest Contiguous Subsequence, 3-SUM Problem
    \end{itemize}
    \item Inductive Algorithms
    \begin{itemize}
        \item Polynomial Evaluation, One-to-One Mapping, Celebrity Problem
    \end{itemize}
    \item Divide and Conquer
    \begin{itemize}
        \item Exponentiation, Recurrences, Closest Pair
        \item Strassen’s Matrix Multiplication, Fast Fourier Transform
    \end{itemize}
    \item Greedy Algorithms
    \begin{itemize}
        \item Coin Change, Scheduling Problems, Fractional Knapsack
        \item Huffman Coding, Stable Matching (Gale-Shapley)
    \end{itemize}
    \item Dynamic Programming
    \begin{itemize}
        \item Fibonacci, Weighted Interval Scheduling, Coin Change
        \item Matrix Chain Multiplication, Knapsack, Sequence Alignment
        \item Longest Common Subsequence, Longest Increasing Subsequence
        \item Maximum Weight Independent Set on Trees, Optimal BST
    \end{itemize}
    \item State Space Search
    \begin{itemize}
        \item Backtracking: Eight Queens, Subset Sum
        \item Branch and Bound: TSP, Game Trees, Alpha-Beta Pruning
    \end{itemize}
    \item Graph Algorithms
    \begin{itemize}
        \item MST: Kruskal and Prim
        \item Fibonacci Heaps, Amortized Analysis for Decrease-Key
        \item All-Pairs Shortest Paths: Floyd-Warshall, Johnson
    \end{itemize}
    \item String Matching
    \begin{itemize}
        \item Fingerprinting and Rabin-Karp
        \item Finite Automata and Knuth-Morris-Pratt (KMP)
    \end{itemize}
    \item Network Flows
    \begin{itemize}
        \item Max-Flow and Min-Cut: Ford-Fulkerson
        \item Improvements: Edmonds-Karp
        \item Applications: Bipartite Matching, Disjoint Paths, Matrix Rounding
    \end{itemize}
    \item Linear Programming
    \begin{itemize}
        \item Standard Formulation, Modeling with LP
        \item Simplex Algorithm
    \end{itemize}
    \item Computational Complexity
    \begin{itemize}
        \item Polynomial-Time Reductions, Satisfiability
        \item NP Class, NP-Completeness, Cook’s Theorem
        \item Hamiltonian Cycle, Graph Coloring, Subset Sum
    \end{itemize}
    \item Approximation Algorithms
    \begin{itemize}
        \item Vertex Cover, TSP, Approximation Hardness
        \item PTAS, Knapsack Problem
    \end{itemize}
\end{itemize}

\textbf{Books Used:}
\begin{itemize}
    \item J. Kleinberg and E. Tardos, \textit{Algorithm Design}, Addison Wesley, 2005.
    \item T. Cormen, C. Leiserson, R. Rivest, and C. Stein, \textit{Introduction to Algorithms}, 3rd Edition, MIT Press, 2009.
    \item U. Manber, \textit{Introduction to Algorithms: A Creative Approach}, Addison-Wesley, 1989.
    \item G. Brassard, P. Bratley, \textit{Algorithmics: Theory and Practice}, Prentice-Hall, 1988.
\end{itemize}

\textbf{Evaluation:}
\begin{itemize}
    \item Three Theoretical Assignments: 3 points
    \item Three Programming Assignments: 3 points
    \item Midterm Exam: 7 points
    \item Final Exam: 7 points
    \item ACM-style Programming Contest: 1+ bonus point
\end{itemize}

\newpage

\section{Industrial Automation Lab}
\subsection*{Course Code: 40401, Units: 1, Prerequisite: Computerized Measurement and Control, Co-requisite: None}

\textbf{Course Summary:} 
\begin{quote}
   This laboratory course aims to provide students with practical familiarity with tools, equipment, and software used in industrial automation. Students will be able to understand and, if necessary, design, implement, or improve automated processes in production, assembly, packaging, monitoring, and quality control environments.
\end{quote}

\textbf{Course Outline:}
\begin{itemize}
    \item Actuators such as various motors, servomotors, drivers, valves, relays, and switches
    \item Installation and setup of LabView software and introduction to its environment
    \item Building and testing a TCP connection in LabView
    \item Designing and implementing a three-floor elevator simulation in LabView
    \item Introduction to LogoSoft software
    \item Familiarization with one or more commonly used industrial Programmable Logic Controllers (PLC) and ladder logic programming; use of analog and digital I/O interface boards based on industrial PCs
    \item Implementation of a four-phase traffic light system using a traffic light training board and PLC
    \item Designing and implementing a tank mixer system on the training board and PLC
    \item Designing and implementing an elevator system in LogoSoft
    \item Higher-level automation programming using Step 7 or Grafcet programming paradigms
\end{itemize}

\textbf{Books Used:}
\begin{itemize}
    \item G. Dunning, \textit{Introduction to Programmable Logic Controller}, 3rd Edition, Thompson, 2017.
    \item F. D. Petruzella, \textit{Programmable Logic Controllers}, McGraw-Hill Education, 5th Edition, 2016.
    \item C. T. Jones, \textit{STEP 7 programming made easy in LAD, FBD, and STL: A practical guide to programming S7-300/S7-400 Programmable Logic Controllers}, Patrick-Turner Publishing, 2013.
    \item R. D. Rosandich, \textit{Fundamentals of Ladder Diagram Programming}, EC \& M Books, 1999.
    \item J. Ravis and J. Kring, \textit{LabVIEW for Everyone: Graphical Programming Made Easy and Fun}, 3rd Edition, Prentice Hall, 2006.
\end{itemize}

\textbf{Evaluation:}
(Not specified)

\newpage

\section{VLSI Lab}
\subsection*{Course Code: 40402, Units: 1, Prerequisite: None, Co-requisite: VLSI Design}

\textbf{Course Summary:} 
\begin{quote}
   This laboratory course aims to familiarize students with automated tools for designing and analyzing digital chips. Students will apply the concepts learned in the VLSI Design course using these tools for practical experiments.
\end{quote}

\textbf{Course Outline:}
\begin{itemize}
    \item Implementation of an inverter and analysis of its circuit characteristics
    \item Design and simulation of a 4-bit counter using HSpice
    \item Implementation of a NOR3 gate in three logics: Static CMOS, Pseudo-NMOS, and Domino Logic, and their comparison
    \item Gate sizing to optimize the speed of a path
    \item Designing a gate using layout drawing tools and verifying its correctness
    \item Introduction to the synthesis tool Design Compiler and synthesis of a 16-bit multiplier for speed and area optimization
    \item Power consumption calculation using Power Compiler tool and applying Clock Gating; comparison with previous power consumption
    \item Introduction to automatic layout design tool SOC Encounter and layout design of a simple sequential circuit
    \item Layout design of the circuit
    \item Automatic layout design of a 16-bit multiplier and functional verification using Modelsim and Hsim
\end{itemize}

\textbf{Books Used:}
\begin{itemize}
    \item Wayne Wolf, \textit{Modern VLSI Design: IP-Based Design}, 4th Edition, Prentice-Hall, 2009.
\end{itemize}

\textbf{Evaluation:}
(Not specified)

\newpage

\section{Biology Laboratory}
\subsection*{Course Code: 40409, Units: 1, Prerequisite: None, Co-requisite: None}

\textbf{Course Summary:} 
\begin{quote}
   This course covers the principles and methods of molecular and cellular biology techniques. Students will gain practical experience studying cell structure, organelles, and various methods to examine cellular behavior and processes.
\end{quote}

\textbf{Course Outline:}
\begin{itemize}
    \item Introduction to skills in molecular and cellular biology laboratory
    \item Study of the function and components of the optical microscope, structural overview, general microscopy techniques, applications, and functions of research and advanced microscopes
    \item Examination of cell structure and function in samples of unicellular organisms and animal and plant cells (using optical microscope)
    \item Measurement of cell dimensions and microscopic samples
    \item Cell counting in suspension media
    \item Cellular staining and organelles such as mitochondria and lysosomes in cells
    \item Study of mitosis cell division process and observation of its stages
    \item Familiarization with permanent slide preparation from plant and animal tissues
    \item General staining of nucleus and cytoplasm
    \item Plasmid DNA isolation, PCR, sequence result analysis, cell culture database searching, DNA transfection into cells, and gene expression study
\end{itemize}

\textbf{Books Used:}
(Not specified)

\newpage

\section{Advanced Logic Design}
\subsection*{Course Code: 40412, Units: 3, Prerequisite: Logic Circuits, Co-requisite: None}

\textbf{Course Summary:} 
\begin{quote}
   This course introduces students to the concepts of synchronous and asynchronous circuit design and timing hazards, with a focus on advanced digital circuit design considerations such as testability and power consumption.
\end{quote}

\textbf{Course Outline:}
\begin{itemize}
    \item Review of sequential circuits
    \item Design and simplification of synchronous sequential circuits
    \item Asynchronous sequential circuits
    \item Timing delays and types of hazards
    \item Multivalued and mixed logic
    \item Design considerations for testability and low power in modern designs
\end{itemize}

\textbf{Evaluation:}
\begin{itemize}
    \item Theoretical Exercises: 3 points
    \item Mid-term and Final Exams: 15 points
    \item Quizzes: 2 points
\end{itemize}

\textbf{Books Used:}
\begin{itemize}
    \item B. J. LaMeres, \textit{Introduction to Logic Circuits \& Logic Design with VHDL}, 2nd Edition, Springer, 2019.
    \item T. Ndjountche, \textit{Digital Electronics Vol. 2, Sequential and Arithmetic Logic Circuits}, Wiley, 2016.
    \item Ch. H. Roth and L. L. Kinney, \textit{Fundamentals of Logic Design}, 7th Edition, Cengage Learning, 2013.
    \item V. G. Oklobdzija, V. M. Stojanovic, D. M. Markovic, and N. M. Nedovic, \textit{Digital System Clocking: High-Performance and Low-Power Aspects}, Wiley, 2003.
    \item J. F. Wakerly, \textit{Digital Design Principles \& Practices}, Prentice Hall, 2001.
    \item M. M. Mano, Ch. R. Kime, and T. Martin, \textit{Logic \& Computer Design Fundamentals}, 5th Edition, Prentice Hall, 2006.
    \item Alireza Ejlali, \textit{Logic Circuits}, 1st Edition, Nasir Publishing, 2018.
\end{itemize}

\newpage

\section{Web Programming}
\subsection*{Course Code: 40419, Units: 3, Prerequisite: Advanced Programming, Co-requisite: None}

\textbf{Course Summary:} 
\begin{quote}
   This course aims to familiarize students with the fundamental concepts and principles of web software design. Students will learn about client-side and server-side programming and their interaction, as well as become acquainted with a widely used framework for web application development.
\end{quote}

\textbf{Course Outline:}
\begin{itemize}
    \item Introduction (1 session)
    \begin{itemize}
        \item Course overview, web history, HTTP protocol
    \end{itemize}
    \item Page Design (2 sessions)
    \begin{itemize}
        \item HTML structure, elements and attributes, paragraphs, formatting, links, lists
        \item Images, tables, forms, new HTML5 elements
    \end{itemize}
    \item Styling (2 sessions)
    \begin{itemize}
        \item CSS definition, formatting, selectors, inheritance and cascade, design principles
        \item Page layout, box model, floats, positioning, pseudo-classes
    \end{itemize}
    \item JavaScript (4 sessions)
    \begin{itemize}
        \item Language structure, uses, statements and functions, variables and data types, control structures
        \item Arrays, objects, object definitions, constructors, data encapsulation
        \item DOM model, editing elements and styles, event handling, exceptions
        \item jQuery library, selectors, events, effects and animations
    \end{itemize}
    \item Data Storage (2 sessions)
    \begin{itemize}
        \item Introduction to XML, uses, DTD, XSLT transformation, introduction to JSON
        \item Relational databases, database creation, SQL query language
    \end{itemize}
    \item Server Interaction (2 sessions)
    \begin{itemize}
        \item CGI interface, GET and POST submission, form processing, cookies
        \item AJAX usage, request sending, response parsing, applications
    \end{itemize}
    \item Python (5 sessions)
    \begin{itemize}
        \item Language structure, operators, data types, lists, strings, tuples, dictionaries
        \item Functions, modules, packages, anonymous functions, variable arguments, decorators
        \item Classes and objects, constructors, inheritance, exception handling
        \item Files, text processing, regular expressions, applications
        \item Web page reading, Python web server, introduction to WSGI
    \end{itemize}
    \item Web Architecture (2 sessions)
    \begin{itemize}
        \item Layering, client-server architecture, three-tier architecture, MVC architecture
        \item Data models, types of relations, mapping to relational databases
    \end{itemize}
    \item Django Framework (6 sessions)
    \begin{itemize}
        \item Basic concepts, installation and setup, components, overall architecture
        \item Project creation, database definition, admin setup, adding views
        \item Model layer, object-relational mapping, inheritance, query execution
        \item View layer, URL mapping, request and response objects, generic views
        \item Template layer, template language, built-in tags and filters
        \item Form processing, built-in widgets, validation
    \end{itemize}
    \item Advanced Topics (4 sessions, if time permits)
    \begin{itemize}
        \item Middleware, optimization, compression, caching usage
        \item Authentication, access control, user and group management
        \item Security, protection against attacks, encryption
        \item Sessions, session state storage, hybrid methods
        \item Internationalization, localization, translation tools
    \end{itemize}
\end{itemize}

\textbf{Evaluation:}
\begin{itemize}
    \item Practical Exercises: 5 points
    \item Project: 5 points
    \item Midterm Exam: 4 points
    \item Final Exam: 6 points
\end{itemize}

\textbf{Books Used:}
\begin{itemize}
    \item S. M. Schafer, \textit{HTML, XHTML, and CSS Bible}, 5th Edition, Wiley Publishing, 2010.
    \item J. Forcier, P. Bissex, and W. Chun, \textit{Python Web Development with Django}, Pearson Addison-Wesley, 2009.
    \item W. J. Chun, \textit{Core Python Applications Programming}, 3rd Edition, Pearson Addison-Wesley, 2012.
    \item M. Fowler, D. Rice, M. Foemmel, E. Hieatt, R. Mee, and R. Stafford, \textit{Patterns of Enterprise Application Architecture}, Pearson Addison-Wesley, 2003.
\end{itemize}

\newpage

\section{Information Technology Project Management}
\subsection*{Course Code: 40428, Units: 3, Prerequisite: None, Co-requisite: None}

\textbf{Course Summary:}
\begin{quote}
IT professionals in managerial and executive roles deal with projects that combine software, hardware, communication, and information components, often involving multiple teams. Managing these projects is challenging and even more complex with outsourcing. This course introduces students to advanced project management concepts specific to IT projects, including software project management within the broader scope of IT project management. Students will learn to function as executors, clients, consultants, or supervisors throughout the project lifecycle. Additionally, the course builds skills in using common project management tools and software to effectively manage IT projects.
\end{quote}

\textbf{Course Outline:}
\begin{itemize}
    \item Introduction (2 sessions)
    \begin{itemize}
        \item Objectives, syllabus overview, teaching model and framework
        \item Fundamental management concepts
    \end{itemize}
    \item Overview of IT Project Management (1 session)
    \item Business Cases (2 sessions)
    \item Project Charter (2 sessions)
    \item Project Team (2 sessions)
    \item Scope Management Plan (2 sessions)
    \item Work Breakdown Structure (2 sessions)
    \item Scheduling and Budgeting Projects (2 sessions)
    \item Project Management Software, Websites, and Dashboards (1 session)
    \item Project Management Body of Knowledge (PMBOK) Standards (1 session)
    \item Risk Management Plan (2 sessions)
    \item Communication Management Plan (2 sessions)
    \item Quality Management for IT Projects (1 session)
    \item Change Management, Resistance, and Conflict Resolution (2 sessions)
    \item Procurement and Outsourcing Management (1 session)
    \item Leadership and Project Etiquette (2 sessions)
    \item Project Implementation and Closure Plan (1 session)
    \item Maturity Models and Agile Methods in IT Project Management (1 session)
\end{itemize}

\textbf{Evaluation:}
\begin{itemize}
    \item Skill Exercises (Simulated Project Management Activities): 6 points
    \item Midterm and Final Exams: 12 points
    \item Quizzes: 2 points
\end{itemize}

\textbf{References:}
\begin{itemize}
    \item Jack T. Marchewka, \textit{Information Technology Project Management}, WILEY, 2014.
\end{itemize}

\newpage

\section{Multicore Computing}
\subsection*{Course Code: 40432, Units: 3, Prerequisites: Advanced Programming, Computer Architecture, Co-requisite: None}

\textbf{Course Objectives:} \\
The main objective of this course is to familiarize students with the architecture of multicore and manycore systems and parallel programming for these systems. The course begins with an overview of architecture, fundamental concepts, and challenges of multicore and manycore systems, followed by introduction to tools and methods for parallel programming on various multicore and manycore platforms.

\textbf{Course Outline:}
\begin{itemize}
    \item Introduction to multicore system architecture and parallel programming models
    \item History of multicore systems
    \item Challenges of efficient programming on multicore systems
    \item Levels of parallelism in programs
    \item Performance acceleration analysis on homogeneous and heterogeneous multicore systems
    \item Examples of real multicore systems
    \item Shared-memory multiprocessors
    \begin{itemize}
        \item Architecture overview
        \item Cache coherence problem and solutions
        \item Programming models and thread synchronization
        \item Handling critical sections
        \item Ideas to improve parallel programs
        \item Common parallel computation and data management patterns
        \item Computational patterns: Map, Reduction, Scan, Stencil, Recurrence, Fork-Join
        \item Data management patterns: Gather, Scatter, Pack, Geometric Decomposition and Partitioning
    \end{itemize}
    \item General parallel programming on multicore systems
    \begin{itemize}
        \item Programming with Pthreads
        \item Programming with OpenMP
    \end{itemize}
    \item Parallel programming on vector systems
    \begin{itemize}
        \item Overview of vector and array systems
        \item Intel SIMD ISA introduction
        \item CELL BE architecture and programming
    \end{itemize}
    \item Parallel programming on general-purpose graphics processors (GPGPU)
    \begin{itemize}
        \item Overview of GPU architecture
        \item NVIDIA GPU architectures overview
        \item CUDA programming
        \item NVIDIA Profiler introduction
    \end{itemize}
    \item Introduction to parallel programming in distributed systems
    \begin{itemize}
        \item MPI library introduction and message-passing programming model
    \end{itemize}
\end{itemize}

\textbf{Evaluation:}
\begin{itemize}
    \item Theoretical Exercises: 3 points
    \item Midterm and Final Exams: 15 points
    \item Quizzes: 2 points
\end{itemize}

\textbf{References:}
\begin{itemize}
    \item D. A. Patterson and J. L. Hennessy, \textit{Computer Architecture: A Quantitative Approach}, Morgan Kaufmann, 2019.
    \item J. Sanders and E. Kandrot, \textit{CUDA by Examples: An Introduction to GPGPU Programming}, Addison-Wesley, 2011.
    \item D. B. Kirk and W. W. Hwu, \textit{Programming Massively Parallel Processors: A Hands-on Approach}, NVIDIA, 2010.
    \item M. McCool, A. D. Robison, and J. Reinders, \textit{Structured Parallel Programming}, Elsevier, 2012.
\end{itemize}

\newpage

\section{Computer Graphics}
\subsection*{Course Code: 40447, Units: 3, Prerequisite: None, Co-requisite: Algorithm Design}

\textbf{Course Objectives:} \\
This course introduces students to fundamental concepts of computer graphics, with a strong emphasis on 3D computer graphics, transformations and projections in 3D, lighting, coloring graphical scenes, and computer games using OpenGL. OpenGL is used within high-level programming languages such as C, C++, and Java. Students are expected to be proficient in at least one of these languages and will learn to use OpenGL throughout the semester.

\textbf{Course Outline:}
\begin{itemize}
    \item Introduction to basic concepts and graphics hardware
    \item 3D geometric transformations
    \item 3D affine transformations
    \item 3D object rendering
    \item Viewing concepts
    \item Steps to produce a graphical scene
    \item Coordinate systems
    \item Projection transformations: perspective, parallel, and oblique
    \item Rendering 3D curved and triangulated surfaces
    \item Introduction to spline functions and their applications
    \item Cubic and quartic spline functions including Bézier, B, Beta, and rational splines
    \item Rendering a spline using other spline functions
    \item Blob objects, axial rendering, and methods based on well-defined geometric shapes
    \item Octrees
    \item Binary space partition trees (BSP)
    \item Visible surface determination methods
    \item Classification, introduction, and comparison of algorithms
    \item Phong lighting model, lighting methods, and surface rendering techniques
    \item Fast rendering algorithms
    \item Texture mapping and surface detailing
    \item Haar models and their applications
    \item Global illumination and shaders
    \item Introduction to fractal geometry for objects and scenes not describable by Euclidean geometry
    \item Data visualization
    \item Computer animation
    \item Traditional animation methods
    \item Animation sequence design
    \item General animation functions
    \item Keyframe systems
    \item Calculating displacements and motion at varying speeds
    \item Camera path calculation
    \item Motion capture techniques for full-body and facial motion and their applications in animation, films, and games
    \item Introduction to computer game development
    \item Main elements including static (background) and moving objects, physics
    \item Texture application on objects
    \item Artificial intelligence, scenarios, game types, and music
    \item Introduction to game engines and their features
    \item Game production management
    \item Testing game development stages and market release
\end{itemize}

\textbf{Evaluation:}
\begin{itemize}
    \item First midterm exam: 2.5 points
    \item Second midterm exam: 2.5 points
    \item Final exam: 5 points
    \item Programming assignments: 10 points
\end{itemize}

\textbf{References:}
\begin{itemize}
    \item Hearn and Baker, \textit{Computer Graphics with OpenGL}, 4th Edition, Prentice Hall, 2011.
    \item Steve Marschner and Peter Shirley, \textit{Fundamentals of Computer Graphics}, 4th Edition, CRC Press, 2016.
    \item Edward Angel, \textit{OpenGL, A Primer}, Addison Wesley, 2002.
\end{itemize}

\newpage

\section{Interface Circuits}
\subsection*{Course Code: 40433, Units: 3, Prerequisite: Computer Architecture, Co-requisite: None}

\textbf{Course Objectives:} \\
This course introduces students to various physical interfaces between computer systems and other systems or real environments (analog or peripheral). Students learn the communication protocols, advantages, disadvantages, applications, and design principles of these interfaces to be able to:
\begin{enumerate}
    \item Gain relative mastery of the design principles of each introduced interface and understand their operation comprehensively.
    \item Choose the correct method of connection between two or more computer systems or between a computer system and its analog peripheral environment, depending on the application environment, to plan for information transfer within or between systems.
    \item Achieve a more complete understanding of a system's architecture, components, and interconnections (e.g., in studying an industrial control system for redesign purposes).
\end{enumerate}

\textbf{Course Outline:}
\begin{itemize}
    \item Fundamental concepts of information exchange
    \item Signal characteristics and transmission lines in computer systems
    \item Bandwidth, data rate, compression, coding, and technology constraints
    \item Principles of serial and parallel interface communication
    \item Synchronous and asynchronous interface circuits
    \item Servicing and addressing methods in interface circuits
    \item Internal system buses
    \item Processor buses and memory devices (Hard and On-board Memory)
    \item Peripheral device buses
    \item Interface circuits in embedded and industrial systems
    \item Inter-system computer buses
    \item USB interface (data transfer)
    \item HDMI interface (user interface)
    \item Overview of wireless interfaces:
    \begin{itemize}
        \item Bluetooth
        \item Wireless USB
        \item Zigbee
    \end{itemize}
    \item Software-hardware interfaces (Device drivers) and embedded/real-time operating system usage
    \item Overview of analog interfacing and ADC/DAC converters, sensors, actuators, electromagnetic interference, crosstalk, grounding, and analog-digital interface design considerations
    \item Practical example of interface circuits based on Raspberry Pi or similar microcontroller boards
\end{itemize}

\textbf{Evaluation:}
\begin{itemize}
    \item Theoretical exercises: 3 points
    \item Midterm and final exams: 15 points
    \item Quizzes: 2 points
\end{itemize}

\textbf{References:}
\begin{itemize}
    \item Jonathan W. Valvano, \textit{Embedded Microcomputer Systems: Real Time Interfacing}, 3rd Edition, Cengage Learning, 2011.
    \item Gourab Sen Gupta and Subhas Chandra Mukhopadhyay, \textit{Embedded Microcontroller Interfacing, Designing Integrated Projects}, Springer, 2010.
    \item Stuart R. Ball, \textit{Analog Interfacing to Embedded Microprocessor Systems}, Elsevier, 2004.
\end{itemize}

\newpage

\section{Theory of Computation}
\subsection*{Course Code: 40455, Units: 3, Prerequisite: Data Structures and Algorithms, Co-requisite: None}

\textbf{Course Objectives:} \\
This course aims to introduce students to the theoretical foundations of computation, main concepts of computability models, solvable problems, mathematical logic, and an introduction to automata theory over infinite string or tree inputs. It provides the theoretical basis for students pursuing graduate studies in computation theory, algorithms, formal methods in software engineering, system verification, and lays the mathematical logic foundation needed for artificial intelligence.

\textbf{Course Outline:}

\begin{enumerate}
    \item \textbf{Computability Theory and Introduction to Computational Complexity}
    \begin{itemize}
        \item Turing machine model, Church-Turing thesis, decidable (recursive) and recognizable (recursively enumerable) languages and functions, uncomputable functions, halting problem, universal Turing machine, multi-tape and nondeterministic Turing machines and their equivalence (3 lectures)
        \item Proof techniques for undecidability and non-recognizability including reduction from the halting problem and many-one reductions (2 lectures)
        \item Introduction to other computational models (2 lectures)
        \begin{itemize}
            \item Random Access Machine (RAM) model (von Neumann architecture)
            \item Kleene recursive functions
            \item Lambda calculus (Church)
            \item Post systems
            \item Recursion theorem and self-reference (1 lecture)
        \end{itemize}
        \item Computational complexity of information and string complexity (2 lectures)
        \item Introduction to complexity theory, overview of time and space complexity classes and hard problems (3 lectures)
    \end{itemize}
    
    \item \textbf{Mathematical Logic from Computation Theory Perspective}
    \begin{itemize}
        \item Propositional logic: syntax, semantics, axiomatic systems, soundness and completeness theorems, decidability of propositional logic (2 lectures)
        \item First-order logic: syntax, semantics, compactness theorem, Löwenheim–Skolem theorem (2 lectures)
        \item Axiomatic system of first-order logic and its soundness theorem (1 lecture)
        \item Gödel's completeness theorem for first-order logic (1 lecture)
        \item Church's theorem on the undecidability of first-order logic (2 lectures)
        \item Axiomatic systems of number theory and Gödel’s incompleteness theorems (both versions) (2 lectures)
    \end{itemize}

    \item \textbf{Introduction to Automata Theory on Infinite Inputs}
    \begin{itemize}
        \item Büchi and Rabin automata on infinite strings (2 lectures)
        \item Complementation and emptiness checking for Büchi automata, nondeterministic Büchi automata, Safra’s theorem (3 lectures)
        \item Relation between decidability problems in logic and automata theory (2 lectures)
        \item Introduction to automata on tree inputs (2 lectures)
    \end{itemize}
\end{enumerate}

\textbf{Evaluation:}
\begin{itemize}
    \item Midterm exam: 25\% of total grade
    \item Final exam: 40\% of total grade
    \item At least six sets of exercises: 25\% of total grade
    \item Continuous assessment including several announced quizzes: 10\% (max 5\% extra credit)
    \item Research report and presentation (optional): up to 15\% bonus points
\end{itemize}

\textbf{References:}
\begin{itemize}
    \item G. Boolos, J. Burgess, and R. Jeffrey, \textit{Computability and Logic}, 5th Edition, Cambridge University Press, 2007.
    \item D. Kozen, \textit{Theory of Computation}, Springer, 2006.
    \item S. Hedman, \textit{A First Course in Logic: An Introduction to Model Theory, Proof Theory, Computability, and Complexity}, Oxford University Press, 2004.
    \item M. Sipser, \textit{Introduction to the Theory of Computation}, 2nd Edition, Thompson, 2006.
\end{itemize}

\newpage

\section{IT Strategic Planning and Management}
\subsection*{Course Code: 40448, Units: 3, Prerequisite: IT Project Management, Co-requisite: None}

\textbf{Course Summary:} 
\begin{quote}
This course provides students with theoretical and practical knowledge of strategic IT management and planning within organizations. It covers selecting appropriate strategic study methods based on organizational capacity, using suitable frameworks, and producing transition solutions through adaptable engineering models. Students will understand the need for architectural roadmaps and their updates to move from current to desired states, enabling integration of solution systems crucial for national projects such as e-government. The course also develops students' skills to extract systemic solutions through practical exercises.
\end{quote}

\textbf{Course Outline:}
\begin{itemize}
    \item Introduction and course framework (2 sessions)
    \item Terminology of strategic management and planning
    \item Comprehensive 360-degree view of traditional strategic planning (4 sessions)
    \item Analytical tools: IFE, EFE, SPACE, SWOT, QSPM (2 sessions)
    \item Developing foundational organizational strategic plans (2 sessions)
    \item Types of strategic studies from business to technology (2 sessions)
    \item Organizational information architecture for managers (3 sessions)
    \item IT strategic planning (2 sessions)
    \item Hanschke’s organizational architecture (2 sessions)
    \item Hanschke’s IT landscape management (2 sessions)
    \item Hanschke’s technical standards for organizational architecture (2 sessions)
    \item Reference models, policies, and change statements (1 session)
    \item Overview of organizational architecture methodologies (1 session)
    \item Königsberg’s IT strategic planning (1 session)
    \item Organizational architecture in Iran: history and national model (1 session)
    \item Organizational architecture maturity models (1 session)
    \item From data governance to architecture governance to digital transformation including ITIL and COBIT (2 sessions)
\end{itemize}

\textbf{Books Used:}
\begin{itemize}
    \item Inge Hanschke, \textit{Strategic IT Management}, Springer, 2010.
    \item Danny Greefhorst and Erik Proper, \textit{Architecture Principles}, Springer, 2011.
    \item Martin Op't Land, \textit{Enterprise Architecture Creating Value by Informed Governance}, Springer, 2009.
    \item Mario Godinez, \textit{The Art of Enterprise Information Architecture}, IBM Press, 2010.
\end{itemize}

\textbf{Evaluation:}
\begin{itemize}
    \item Midterm and final exams: 12 points
    \item Quizzes: 6 points
    \item Individual study of latest technologies: 2 points
\end{itemize}

\newpage

\section{Computer Measurement and Control}
\subsection*{Course Code: 40463, Units: 3, Prerequisite: Fundamentals of Electrical and Electronic Circuits, Co-requisite: None}

\textbf{Course Summary:} 
\begin{quote}
This course aims to familiarize students with various sensors and actuators, interface circuits, amplifiers, voltage level converters for sensor outputs and actuator commands, analog-to-digital and digital-to-analog converters, processor units, and other components of a computer-based (digital) control system.
\end{quote}

\textbf{Course Outline:}
\begin{itemize}
    \item Introduction to control processes (3 sessions)
    \begin{itemize}
        \item Control systems, control process block diagrams, system evaluation, analog and digital processing, units, standards, and definitions, sensor response time, computational accuracy, and statistical quantities
    \end{itemize}
    \item Analog signal shaping (4 sessions)
    \begin{itemize}
        \item Basic principles, passive circuits, operational amplifier circuits
    \end{itemize}
    \item Digital signal shaping (4 sessions)
    \begin{itemize}
        \item Basic principles, converters, data acquisition systems
    \end{itemize}
    \item Temperature sensors (4 sessions)
    \begin{itemize}
        \item Metal resistors, thermistors, thermocouples, other temperature sensors
    \end{itemize}
    \item Mechanical sensors (4 sessions)
    \begin{itemize}
        \item Displacement, position and status sensors, force sensors, motion sensors, pressure sensors, fluid flow sensors
    \end{itemize}
    \item Optical sensors (2 sessions)
    \begin{itemize}
        \item Light intensity detectors, remote thermometry, light sources
    \end{itemize}
    \item Final control elements (3 sessions)
    \begin{itemize}
        \item Final control operations, signal conversion, industrial electronics, actuators, controller components
    \end{itemize}
    \item Discrete state process control (2 sessions)
    \begin{itemize}
        \item Definitions, system characteristics, relay controllers and ladder diagrams, programmable logic controllers
    \end{itemize}
    \item Basics of controllers (1 session)
    \begin{itemize}
        \item Process specifications, control system parameters, discontinuous, continuous, and combined control modes
    \end{itemize}
    \item Analog controllers (1 session)
    \begin{itemize}
        \item General capabilities, electronic controllers, pneumatic controllers
    \end{itemize}
    \item Digital controllers (2 sessions)
    \begin{itemize}
        \item Digital control methods, computer application in process control, digital data characteristics, controller software, examples of computer control
    \end{itemize}
\end{itemize}

\textbf{Books Used:}
\begin{itemize}
    \item Curtis D. Johnson, \textit{Process Control Instrumentation Technology}, 7th Edition, Prentice-Hall International, Inc., 2006.
    \item Alan J. Crispin, \textit{Programmable Logic Controllers and Their Engineering Applications}, McGraw-Hill, 1990.
\end{itemize}

\textbf{Evaluation:}
\begin{itemize}
    \item Theoretical exercises: 4 points
    \item Midterm and final exams: 16 points
\end{itemize}

\newpage

\section{Information Technology}
\subsection*{Course Code: 40467, Units: 3, Prerequisite: None, Co-requisite: None}

\textbf{Course Summary:} 
\begin{quote}
The broad domain of computer applications forms the framework of information technology (IT) discussions. This course introduces students to the principles, definitions, concepts, applications, organizational and social impacts, and managerial concepts of IT along with its foundations and architecture. Since computer and IT engineers are innovators and promoters of new solutions in this field, they must be aware of the latest concepts, achievements, and applications of IT globally and in Iran. The superficial breadth of concepts in this course provides a structural foundation for deeper study in subsequent courses.
\end{quote}

\textbf{Course Outline:}
\begin{itemize}
    \item Introduction (1 session)
    \begin{itemize}
        \item First lecture, values and harms
        \item Differences, similarities, and overlaps among Computer Science, Computer Engineering, Software Engineering, and Information Technology
        \item Information (IT) and information systems in global standards
    \end{itemize}
    \item History, definitions, principles, frameworks, and preconceptions (2 sessions)
    \begin{itemize}
        \item From Wiener to Dreyfus, Toffler, Castells to Freeman's conceptualization
        \item From Cybernetics to Computers, Informatics, and Information Technology
        \item Impact perspective: technology is neither good nor bad but definitely not neutral (Kranberg)
    \end{itemize}
    \item Data, information, and knowledge: definitions, differences, similarities, and IT values (3 sessions)
    \begin{itemize}
        \item Definitions and relations of data, information, and knowledge
        \item Shannon's information theory, Lucien Gérard's value of information
        \item Cycles of data, information, and knowledge and their relationships
        \item Types of information values
        \item IT-based organizations in the digital economy and IT management
    \end{itemize}
    \item Network computing and IT management in digital economy-based organizations (2 sessions)
    \begin{itemize}
        \item Networks, tools, networking contracts, network types, internetworking, and the Internet
        \item Evolution of automation in organizations
        \item Telecommuting and virtual organizations
    \end{itemize}
    \item IT absorption capacity, electronic readiness, digital rankings, criteria, and digital divide (2 sessions)
    \begin{itemize}
        \item Absorptive capacity of technology, calculation and enhancement methods
        \item Digital readiness and digital divide and the use of their measurements
        \item Ranking models, parameters, calculation methods, and their values
        \item Electronic readiness and its computational model
    \end{itemize}
    \item E-commerce, business intelligence, and data warehouses (3 sessions)
    \begin{itemize}
        \item Definitions, differences, and similarities between e-commerce and e-business
        \item Various transactional relationships in e-commerce
        \item Business models in the digital economy
        \item Business intelligence: definitions, applications, and usage
        \item Data warehouses: definitions, architecture, and their role in business intelligence
        \item Types of data mining: data, text, web mining and applications in business intelligence
    \end{itemize}
    \item Wireless, mobile, ubiquitous, pervasive, and value-adding computing (2 sessions)
    \begin{itemize}
        \item Mobile and wireless communications: foundations and applications
        \item Communication and information technologies and realization of ubiquitous computing
        \item Pervasive computing and its requirements
        \item Value-added computing, methods of realization and implementation requirements
    \end{itemize}
    \item Local, enterprise, and international systems: features and integration (2 sessions)
    \begin{itemize}
        \item Enterprise systems and priorities for their preparation
        \item Global and international systems: design requirements and implementation features
        \item Legacy systems: needs and integration solutions
        \item Integration technologies and tools
    \end{itemize}
    \item Management support systems, supply chains, enterprise resource planning, and customer relations (2 sessions)
    \begin{itemize}
        \item Types of management information systems, strategic information, execution and decision support systems
        \item Architecture and features
        \item Applications and constraints
    \end{itemize}
    \item Internet structures, IT foundations and architecture (3 sessions)
    \begin{itemize}
        \item Intranets and extranets
        \item Websites, blogs, social networks to enterprise portals and their types
        \item Framework of an e-commerce architecture
        \item Relationship between architecture and IT foundation in enterprises
    \end{itemize}
    \item Contemporary value-added IT applications (2 sessions)
    \begin{itemize}
        \item Geographic information systems: architecture and applications
        \item Global positioning systems
        \item Workflow management systems
        \item Applications and utilization of remote sensing technology
        \item Telecommuting, its facilities and consequences
    \end{itemize}
    \item Effects, ethics, and security of IT (2 sessions)
    \begin{itemize}
        \item Consequences of widespread IT presence globally
        \item Necessity of IT ethics and manners, and their realization and implementation
        \item Virtual world, second life, and social and cultural consequences
        \item IT security and ways to achieve it
    \end{itemize}
    \item Information society and e-government, electronic services and their foundations (2 sessions)
    \begin{itemize}
        \item Definition of e-government: needs, requirements, and prerequisites for its realization
        \item Information society: features and global realization requirements
        \item E-learning: types, needs, and social impacts
        \item Types and applications of electronic services
    \end{itemize}
    \item National and international outlook of information technology (2 sessions)
    \begin{itemize}
        \item History of IT in Iran
        \item Stakeholders, laws, and governing documents of IT in Iran
        \item IT industry and market in Iran
        \item Electronic banking in Iran
        \item Computer and IT education in Iran
        \item Role of IT parks in technology transfer
        \item Current global status of IT
    \end{itemize}
\end{itemize}

\textbf{Books Used:}
\begin{itemize}
    \item Linda Volonino and Efrain Turban, \textit{Information Technology for Management: Improving Performance in The Digital Economy}, 8th Edition, WILEY, 2011.
    \item Efraim Turban, Dorothy Leidner, Ephraim McLean, and James Wetherbe, \textit{Information Technology for Management: Transforming Organizations in the Digital Economy}, 5th Edition, John Wiley \& Sons Inc., 2006.
    \item E. Turban, R. K. Rainer, and R. E. Potter, \textit{Introduction to Information Technology}, 3rd Edition, WILEY, 2005.
    \item Urs Birchler and Monika Butler, \textit{Information Economics}, Routledge, 2007.
    \item E. W. Martin and C. V. Brown, \textit{Managing Information Technology}, 5th Edition, Prentice Hall, 2004.
    \item K. D. Willett, \textit{Information Assurance Architecture}, CRC, 2008.
    \item Thomas H. Davenport and Laurence Prusak, \textit{Information Ecology: Mastering the Information and Knowledge Environment}, OXFORD University Press, 1997.
\end{itemize}

\textbf{Evaluation:}
\begin{itemize}
    \item Theoretical exercises: 7 points
    \item Midterm and final exams: 11 points
    \item Quizzes: 2 points
\end{itemize}

\newpage

\section{Agile Software Development}
\subsection*{Course Code: 40475, Units: 3, Prerequisite: Systems Analysis and Design, Co-requisite: None}

\textbf{Course Objectives:} \\
This course aims to familiarize undergraduate computer engineering students with advanced concepts, principles, and methods of Agile software system development. After reviewing Agile fundamentals and the XP methodology, students will learn about DSDM and DAD methodologies and use them alongside Agile patterns and practices to develop a software system.

\textbf{Course Outline:}
\begin{itemize}
    \item Introduction: Overview of basic concepts and history of Agile development, Agile Manifesto and principles (1 session)
    \item XP (Extreme Programming) Methodology (2 sessions)
    \item DSDM (Dynamic Systems Development Method)
    \begin{itemize}
        \item General framework, principles, and rules (2 sessions)
        \item Feasibility Phase (1 session)
        \item Foundations Phase (2 sessions)
        \item Evolutionary Development Phase (2 sessions)
        \item Deployment Phase (2 sessions)
        \item Roles, deliverables, and Agile practices (3 sessions)
    \end{itemize}
    \item DAD (Disciplined Agile Delivery)
    \begin{itemize}
        \item General framework (1 session)
        \item Inception Phase (1 session)
        \item Elaboration Phase (2 sessions)
        \item Construction Phase (2 sessions)
        \item Transition Phase (1 session)
        \item Iterative activities and Agile practices (2 sessions)
    \end{itemize}
    \item Agile Practices: Team management, design, and Kanban (3 sessions)
    \item Patterns (3 sessions)
\end{itemize}

\textbf{Evaluation:}
\begin{itemize}
    \item Exams: Midterm and final (60\% of total grade)
    \item Exercises and Project: Exercises as part of a DSDM or DAD project, gradually completed and submitted during the semester (40\% of total grade)
\end{itemize}

\textbf{References:}
\begin{itemize}
    \item D. Wells, \textit{Extreme Programming: A Gentle Introduction}, 2013. \url{http://www.extremeprogramming.org}
    \item DSDM Consortium, \textit{The DSDM Project Framework Handbook}, Agile Business Consortium, 2014. \url{https://www.agilebusiness.org/page/TheDSDMAgileProjectFramework}
    \item S. W. Ambler and M. Lines, \textit{Disciplined Agile Delivery: A Practitioner's Guide to Agile Software Delivery in the Enterprise}, IBM Press, 2012.
    \item Agile Alliance, \textit{Agile 101: Subway Map to Agile Practices}, 2015. \url{https://www.agilealliance.org/agile101/subway-map-to-agile-practices/}
    \item E. Gamma, R. Helm, R. Johnson, and J. Vlissides, \textit{Design Patterns: Elements of Reusable Object-Oriented Software}, Addison-Wesley, 1995.
\end{itemize}

\newpage

\section{Machine Learning}
\subsection*{Course Code: 40477, Units: 3, Prerequisite: Artificial Intelligence, Linear Algebra, Co-requisite: None}

\textbf{Course Summary:} 
\begin{quote}
   This course introduces the concepts of machine learning, providing familiarity with its various branches and important theoretical and practical aspects. Key techniques and algorithms across different branches are discussed. In supervised learning, regression and classification problems, their solutions, and model evaluation methods are covered. For classification, various perspectives and corresponding algorithms are introduced. Unsupervised learning topics include density estimation, unsupervised dimensionality reduction, and clustering. Finally, a brief introduction to reinforcement learning is provided.
\end{quote}

\textbf{Course Outline:}
\begin{itemize}
    \item Introduction to Machine Learning and Review of Probability and Linear Algebra (1 session)
    \item ML and MAP Estimation Methods (1 session)
    \item Regression (3 sessions)
    \begin{itemize}
        \item Linear and Non-linear Regression
        \item Overfitting
        \item Error decomposition into Bias, Variance, and Noise
        \item Regularization
        \item Statistical Regression: Relation of SSE-based objective functions to ML and MAP estimates in regression
    \end{itemize}
    \item Model Evaluation and Tuning (1-2 sessions)
    \begin{itemize}
        \item Validation
        \item Cross-validation
        \item Model Selection
        \item Feature Selection
    \end{itemize}
    \item Classification
    \begin{itemize}
        \item Probabilistic Classifiers (3 sessions)
        \begin{itemize}
            \item Decision Theory and Bayes Optimal Classifier
            \item Discriminative and Generative Probabilistic Classifiers
            \item Binary and Multi-class Logistic Regression, Naïve Bayes
        \end{itemize}
        \item Classification using Discriminant Functions (6 sessions)
        \begin{itemize}
            \item Perceptron
            \item Fisher Linear Discriminant
            \item Support Vector Machines (SVM) and Kernel Methods
            \item Neural Networks
        \end{itemize}
        \item Decision Tree (1 session)
        \begin{itemize}
            \item Information Gain and Entropy
            \item ID-3 Algorithm
            \item Tree Growth Stopping and Pruning
        \end{itemize}
    \end{itemize}
    \item Instance-Based Learning Methods (2 sessions)
    \begin{itemize}
        \item Non-parametric Density Estimation
        \item k-Nearest Neighbors Classifier
        \item Locally Weighted Linear Regression
    \end{itemize}
    \item Computational Learning Theory (2 sessions)
    \begin{itemize}
        \item PAC Learning
        \item VC Dimension
        \item Structural Risk Minimization
    \end{itemize}
    \item Ensemble Learning (2 sessions)
    \begin{itemize}
        \item Boosting and Bagging
        \item AdaBoost
    \end{itemize}
    \item Unsupervised Dimensionality Reduction (2 sessions)
    \begin{itemize}
        \item Principal Component Analysis (PCA)
        \item Independent Component Analysis (ICA)
    \end{itemize}
    \item Clustering (3 sessions)
    \begin{itemize}
        \item Partitional Methods (k-means, EM + GMM)
        \item Hierarchical Methods
    \end{itemize}
    \item Reinforcement Learning (2 sessions)
    \begin{itemize}
        \item Markov Decision Processes (MDP)
        \item Model-based Learning Methods
        \item Value Iteration and Policy Iteration
        \item Model-free Learning Methods
        \item SARSA, Q-learning, Temporal Difference Algorithms
    \end{itemize}
    \item Advanced Topics in Machine Learning
\end{itemize}

\textbf{Evaluation:}
\begin{itemize}
    \item Exercises: 20\%
    \item Midterm: 25\%
    \item Final Exam: 35\%
    \item Quizzes: 10\%
    \item Project: 10\%
\end{itemize}

\textbf{References:}
\begin{itemize}
    \item C. Bishop, \textit{Pattern Recognition and Machine Learning}, Springer, 2006.
    \item T. Mitchell, \textit{Machine Learning}, MIT Press, 1998.
    \item K. Murphy, \textit{Machine Learning: A Probabilistic Perspective}, MIT Press, 2012.
    \item T. Hastie, R. Tibshirani, and J. Friedman, \textit{The Elements of Statistical Learning}, 2nd Edition, 2008.
\end{itemize}

\newpage

\section{Application Engineering}
\subsection*{Course Code: 40478, Units: 3, Prerequisite: None, Co-requisite: Systems Analysis and Design}

\textbf{Course Summary:} 
\begin{quote}
   The main goal of this course is to connect students' knowledge from abstract IT subjects such as Strategic Management and E-Commerce with operational courses like Databases, Networks, and Programming. Secondary objectives include familiarity with components of IT solutions and methodologies for system creation by combining these components; understanding common systems and their application domains such as ERP, CRM, and Portals; acquaintance with middleware and platforms usable in designing IT solutions; concepts of modern system production and current technologies; and approaches to handling legacy systems in organizations. The targeted organizations are large, distributed entities requiring more complex and distributed IT solutions.
\end{quote}

\textbf{Course Outline:}
\begin{itemize}
    \item Introduction (3 sessions)
    \begin{itemize}
        \item Organizational Strategies
        \item Common Business Systems
        \item Distributed Organizations and Systems
    \end{itemize}
    \item Application Systems (7 sessions)
    \begin{itemize}
        \item Definition of Application Systems
        \item Common Application Systems such as ERP, CRM, Portal
        \item Relationship between Application Systems and Organizational Strategies
        \item Organizational Process Modeling
        \item Identification of Application Systems based on Organizational Processes
        \item Methodology for Identifying Application Systems
    \end{itemize}
    \item Architecture (7 sessions)
    \begin{itemize}
        \item Software Architecture
        \item Data Architecture
        \item Solution Architecture
    \end{itemize}
    \item Systems Integration (8 sessions)
    \begin{itemize}
        \item Approaches to Legacy Systems in Organizations
        \item Integration of Systems with Each Other or with Legacy Systems
        \item Data Warehousing and its Use for Integration
        \item Strategies for Replacing or Renovating Legacy Systems
        \item Reengineering Patterns
    \end{itemize}
    \item Middleware and Modern Technologies for System Interaction (5 sessions)
    \begin{itemize}
        \item Service-Oriented Architecture (SOA)
        \item Web Services, CORBA, J2EE, etc.
        \item Distributed Transaction Management
        \item Asynchronous Message Exchange
    \end{itemize}
\end{itemize}

\textbf{Evaluation:}
\begin{itemize}
    \item Theoretical and Practical Exercises: 3 points
    \item Midterm and Final Exams: 15 points
    \item Quizzes: 2 points
\end{itemize}

\textbf{References:}
\begin{itemize}
    \item Amjad Umar, \textit{Enterprise Architectures and Integration with SOA – Concepts, Methodology and a Toolset}, NGE Solutions, 2010.
    \item Amjad Umar, \textit{e-Business and Distributed Systems Handbook (from Strategies to Working Solutions)}, NGE Solutions, 2003.
    \item Hans-Erik Eriksson and Magnus Penker, \textit{Business Modeling with UML}, 2000.
\end{itemize}

\newpage

\section{Hardware Description Languages}
\subsection*{Course Code: 40483, Units: 3, Prerequisite: Digital Systems Design, Computer Architecture, Co-requisite: None}

\textbf{Course Summary:} 
\begin{quote}
   This course aims to familiarize students with the required features of hardware description languages (HDLs) compared to software languages. It covers an overview and introduction to three well-known hardware design languages: VHDL, Verilog, and SystemC. Students will work with these languages and understand the important differences among them in hardware modeling. Additionally, the course explains the differences between hardware modeling and system modeling using SystemC.
\end{quote}

\textbf{Course Outline:}
\begin{itemize}
    \item SystemC Language and Hardware Modeling with It
    \begin{itemize}
        \item History and evolution of SystemC
        \item Modules and their components
        \item Ports and their types, concept and applications of signals
        \item Types of processes in SystemC and their uses
        \item Data types in SystemC: two-valued logic, four-valued logic, computational data types, bitwise data types
        \item Complex data types, defining and using structs for signals and ports
        \item Implementation methods of combinational and sequential circuits with SystemC and differences between process types
        \item State machines and types of Mealy and Moore machines, implicit and explicit state machine descriptions
        \item Synthesis of SystemC models: combinational circuit synthesis, important notes to generate the desired circuit, preventing latch generation, sequential circuit synthesis and recommended style
        \item Finite State Machine with Datapath (FSMD) model and its importance, implementation with SystemC
    \end{itemize}
    \item VHDL Language and Hardware Modeling with It
    \begin{itemize}
        \item History, evolution, and strengths of VHDL
        \item Overview of language structure
        \item Delay types in VHDL
        \item Structural description, port connection methods, flip-flop design example, repetitive structure design examples
        \item Parameterizing design and defining configurations
        \item Data types, arrays, physical data
        \item Multi-valued logic and related IEEE packages
        \item Process statements and state machine design
        \item Synthesizable subset and design styles
    \end{itemize}
    \item Brief Review of Verilog and Qualitative Comparison of SystemC, VHDL, and Verilog
\end{itemize}

\textbf{Evaluation:}
\begin{itemize}
    \item Theoretical Exercises: 3 points
    \item Midterm and Final Exams: 15 points
    \item Quizzes: 2 points
\end{itemize}

\textbf{References:}
\begin{itemize}
    \item SystemC User’s Guide, Version 2.0, SystemC Consortium, 2002.
    \item J. Bhaskar, \textit{A SystemC Primer}, Star Galaxy Publishing, 2002.
    \item Peter J. Ashenden, \textit{The Designer's Guide to VHDL}, Elsevier (Morgan Kaufmann), 2008.
    \item Z. Navabi, \textit{VHDL: Analysis and Modeling of Digital Systems}, McGraw Hill, 1998.
    \item D. L. Perry, \textit{VHDL: Programming by Examples}, McGraw Hill, 2002.
\end{itemize}

\newpage

\section{Object-Oriented Systems Design}
\subsection*{Course Code: 40484, Units: 3, Prerequisite: Systems Analysis and Design, Co-requisite: None}

\textbf{Course Objectives:} 
\begin{quote}
   This course introduces undergraduate software engineering students to the concepts, principles, and methods of object-oriented analysis and design of software systems. Students gain a thorough understanding of a modern (third-generation) object-oriented analysis and design methodology and become familiar with GoF design patterns and their practical application.
\end{quote}

\textbf{Course Content:}

\begin{itemize}
    \item Introduction - Overview of Object-Orientation and Evolutionary History of Object-Oriented Analysis and Design (1 session)
    \item Review of Unified Modeling Language (UML) (4 sessions)
    \item USDP Phases and Workflows
    \item The Four Phases (3 sessions)
    \item Requirements Workflow - Identifying and Modeling Use Cases (3 sessions)
    \item Analysis Workflow
    \begin{itemize}
        \item Identifying and Modeling Analysis Objects and Classes (2 sessions)
        \item Identifying and Modeling Relationships Between Analysis Objects and Classes (2 sessions)
        \item Analysis Packages (1 session)
        \item Realizing Use Cases in Analysis (2 sessions)
        \item Activity Modeling (2 sessions)
    \end{itemize}
    \item Design Workflow
    \begin{itemize}
        \item Identifying and Modeling Design Objects and Classes (1 session)
        \item Refining Relationships (1 session)
        \item Interfaces and Components (1 session)
        \item Realizing Use Cases in Design (1 session)
    \end{itemize}
    \item Implementation Workflow (1 session)
    \item Design Patterns
    \begin{itemize}
        \item Design Principles and Rules: The Six Fundamental Principles, GRASP Patterns, Contract-Based Design (1 session)
        \item GoF Design Patterns
        \begin{itemize}
            \item Creational Patterns: Factory Method, Abstract Factory, Builder, Prototype, Singleton (1 session)
            \item Structural Patterns: Adapter, Bridge, Composite, Decorator, Facade, Proxy (1 session)
            \item Behavioral Patterns: Chain of Responsibility, Iterator, Mediator, Memento, Observer, State, Strategy, Visitor (2 sessions)
        \end{itemize}
    \end{itemize}
\end{itemize}

\textbf{Assessment:}
\begin{itemize}
    \item Exams: Midterm and Final Exams (60\% of total grade)
    \item Exercises and Project: Exercises integrated into a course project on analysis and design, progressively developed and submitted during the semester (40\% of total grade)
\end{itemize}

\textbf{References:}
\begin{itemize}
    \item J. Arlow and I. Neustadt, \textit{UML 2 and the Unified Process}, 2nd Edition, Addison-Wesley, 2005.
    \item H. Gomaa, \textit{Software Modeling and Design: UML, Use Cases, Patterns, and Software Architectures}, Cambridge University Press, 2011.
    \item G. Booch, R. A. Maksimchuk, M. W. Engel, B. J. Young, J. Conallen, and K. A. Houston, \textit{Object-Oriented Analysis and Design with Applications}, 3rd Edition, Addison-Wesley, 2007.
    \item E. Gamma, R. Helm, R. Johnson, and J. Vlissides, \textit{Design Patterns: Elements of Reusable Object-Oriented Software}, Addison-Wesley, 1995.
    \item C. Larman, \textit{Applying UML and Patterns: An Introduction to Object-Oriented Analysis and Design and Iterative Development}, 3rd Edition, Prentice-Hall, 2004.
\end{itemize}

\newpage

\section{Introduction to Bioinformatics}
\subsection*{Course Code: 40494, Units: 3, Prerequisite: Data Structures and Algorithms, Engineering Probability and Statistics, Co-requisite: None}

\textbf{Course Objectives:} 
\begin{quote}
   This course aims to familiarize students with the essentials of bioinformatics data analysis. These essentials include a review of key topics in cellular and molecular biology, fundamental bioinformatics algorithms, statistical methods and machine learning techniques used in biomedical data analysis, bioinformatics databases, and practical data analysis using Linux OS and the R programming environment. Upon completion, students are expected to have the foundational knowledge needed to study advanced research and other courses in this field.
\end{quote}

\textbf{Course Content:}

\begin{itemize}
    \item Introduction
    \begin{itemize}
        \item Necessity of Learning Bioinformatics
        \item Applications of Bioinformatics in Biological and Medical Research
    \end{itemize}
    \item Essentials of Cellular and Molecular Biology
    \begin{itemize}
        \item Cell Components
        \item DNA Structure and Replication
        \item RNA Structure and Transcription
        \item Protein Structure and Translation
        \item Gene Expression Regulation
        \item Cellular Differentiation
    \end{itemize}
    \item Introduction to Bioinformatics Data
    \begin{itemize}
        \item Technologies for Biological Data Generation: Microarray and Next Generation Sequencing
        \item Important Biomedical Data Repositories
        \item Integration of Various Databases
    \end{itemize}
    \item Introduction to Statistical Methods
    \begin{itemize}
        \item Gene Expression Difference Analysis
        \item Statistical Tests
        \item p-value and Its Correction Methods
        \item Dimensionality Reduction in Biological Data
    \end{itemize}
    \item Basic Biological Data Analysis Using R
    \begin{itemize}
        \item Introduction to R Programming
        \item Data Visualization in R
        \item R/Bioconductor Libraries
        \item Microarray Gene Expression Data Analysis
        \item RNASeq Data Analysis
        \item GO and Pathway Analysis
        \item GSEA Analysis
        \item Meta-Analysis of Data
    \end{itemize}
    \item Introduction to Data Analysis on Linux Server
    \begin{itemize}
        \item SSH Connection and Secure File Transfer
        \item Bash Programming in Linux Environment
        \item Direct Installation and Use of Bioinformatics Software
        \item Introduction to BioConda
        \item Parallel Execution of Software
    \end{itemize}
    \item Introduction to Bioinformatics Algorithms
    \begin{itemize}
        \item Biological Sequence Alignment
        \item Phylogenetic Trees
        \item Genome Assembly
        \item Read Mapping (Alignment)
        \item Motif Finding
    \end{itemize}
    \item Systems Biology Analyses
    \begin{itemize}
        \item Differential Equations Applications
        \item Differentiation Analysis
        \item Structural Data Analysis
        \item RNA and Protein Folding Problems
        \item Protein-Protein Interaction
    \end{itemize}
\end{itemize}

\textbf{Assessment:}
\begin{itemize}
    \item Workshop: 2 points
    \item Exercises: 5 points
    \item Project: 3 points
    \item Midterm Exam: 5 points
    \item Final Exam: 5 points
\end{itemize}

\textbf{References:}
\begin{itemize}
    \item Bruce Alberts et al., \textit{Essential Cell Biology}, Garland Science, 2013.
    \item Neil C. Jones and Pavel A. Pevzner, \textit{An Introduction to Bioinformatics Algorithms}, The MIT Press, 2004.
\end{itemize}

\newpage

\section{Genetic and Evolution}
\subsection*{Course Code: 40496, Units: 3, Prerequisite: None, Co-requisite: None}

\textbf{Course Objectives:}  
\begin{quote}
   This course introduces students first to the principles of genetics at the cellular, molecular, and organismal levels, and then to the principles of genetics and evolution at larger scales, including populations and communities.
\end{quote}

\textbf{Course Content:}

\begin{itemize}
    \item Introduction to Genetics
    \item Genetics at the Cellular Level
    \begin{itemize}
        \item Basics of Heredity
        \item Revisiting and Extending Heredity Concepts
        \item Genetic Linkage, Recombination, and Genetic Mapping
        \item Genetics of Bacteria and Viruses
        \item Chemical Identity of Genes
        \item DNA Replication and Recombination
        \item Chromosomal and Gene Mutations
    \end{itemize}

    \item Introduction to Evolutionary Theory
    \begin{itemize}
        \item Phylogenetic Nomenclature Code
        \item Genotype-Phenotype Distinction
        \item Definition of Life and Artificial Life
        \item Population Genetics
        \item Population and Unit of Natural Selection
        \item Variation and Heredity
        \item Fisher’s Fundamental Theorem of Natural Selection
        \item Fitness
        \item DNA and RNA Molecules
        \item Replication Fidelity and Mutation
        \item Genetic Drift in Finite Populations
        \item Criticism of Neo-Darwinism
        \item Central Dogma
        \item Age of the Universe
        \item Paleontological Evidence
        \item Lamarck and Weismann
        \item Observations from Domesticated Animals
        \item Elements of Evolution by Natural Selection
        \item Long-term Natural Selection Experiments
        \item Artificial Selection
        \item Main Evidence for Evolution
        \item Population Size and Growth Models
        \item Chaos Theory
        \item Evolution in Diploid Populations
        \item Diploid Organisms
        \item Hardy–Weinberg Principle
        \item Spread of a Beneficial Gene
        \item Dominant and Recessive Genes
        \item Types in Natural Populations
        \item Evidence for Genetic Variation
        \item Nature of Mutation
        \item Mutation-Selection Balance
        \item Mutation Load
        \item Genetic Recombination (Cross Over)
        \item Frequency-dependent Natural Selection
        \item Variable Environments
        \item Evolution at Multiple Loci
        \item Linkage Disequilibrium
        \item Mimicry in Butterflies
        \item Natural Selection and Linkage Disequilibrium
        \item Small and Structured Populations
        \item Population Structure
        \item Inbreeding
        \item Genetic Drift and Mutation
        \item Neutral Evolution and Molecular Evolution Rates
        \item Migration and Population Differentiation
    \end{itemize}
\end{itemize}

\textbf{References:}
\begin{itemize}
    \item Benjamin A. Pierce, \textit{Genetics: A Conceptual Approach}, 5th Edition.
    \item Terry Brown, \textit{Introduction to Genetics: A Molecular Approach}.
    \item John Maynard Smith, \textit{Evolutionary Genetics}, OUP, 1999.
    \item Charlesworth, B. \& Charlesworth, D., “Population Genetics from 1966 to 2016,” 2016.
    \item Singh \& Krimbas, \textit{Evolutionary Genetics}, CUP, 1999.
    \item Charles W. Fox and Jason B. Wolf, \textit{Evolutionary Genetics: Concepts and Case Studies}, Oxford University Press, 2006.
    \item Joseph Felsenstein, \textit{Theoretical Evolutionary Genetics}, 2016.
\end{itemize}

\newpage

\section{Cellular and Molecular Biology}
\subsection*{Course Code: 40495, Units: 3, Prerequisite: None, Co-requisite: None}

\textbf{Course Objectives:}  
\begin{quote}
   This course introduces students to the fundamentals of cellular and molecular biology. It begins with the chemical bases of cellular biomolecules including proteins and nucleic acids, then discusses molecular processes such as DNA replication, transcription, and protein translation. Mechanisms controlling molecular processes in the cell are also covered.
\end{quote}

\textbf{Course Content:}

\begin{itemize}
    \item Chemical and Molecular Foundations of Biology
    \begin{itemize}
        \item Molecules of Life, Genome, Cell Architecture and Function, Cell and Tissue
        \item Chemical Foundations: Covalent bonds, Non-covalent interactions, Chemical components of the cell, Chemical reactions and chemical equilibrium, Bioenergetics
        \item Protein Structure and Function: Hierarchical protein structure, Protein folding, Protein binding and enzyme catalysis, Regulation of protein function, Protein purification, tracing and characterization, Proteomics
    \end{itemize}
    
    \item Genetics and Molecular Biology
    \begin{itemize}
        \item Molecular basis of genetic mechanisms: Structure of nucleic acids, Transcription of protein-coding genes, Decoding of messenger RNA, Protein synthesis on ribosomes, DNA replication, DNA repair and recombination, Viruses
        \item Molecular genetic methods: Genetic mutation analysis for gene study, DNA cloning, Cloning for gene expression studies, Identification of human diseases, Specific gene inactivation
        \item Gene, Genome, and Chromosome: Eukaryotic gene structure, Chromosome organization and non-coding DNA, Mobile elements in genome, Genomics and gene expression
        \item Transcriptional control of gene expression: Gene regulation in bacteria, Gene regulation in eukaryotes, Promoters, RNA polymerase II and transcription factors, Activation and repression mechanisms, Transcription factor activity control, Epigenetic transcription control
    \end{itemize}

    \item Cell Structure and Function
    \begin{itemize}
        \item Cell growth and observation, cell perturbation: Cell culture, Light microscopy and cell structure exploration, Protein localization in cells, Electron microscopy, Cell organelle isolation and study, Cell perturbation and functional studies
    \end{itemize}
\end{itemize}

\textbf{Assessment:}
\begin{itemize}
    \item Exercises and Projects: 30\%
    \item Midterm and Final Exams: 70\%
\end{itemize}

\textbf{References:}
\begin{itemize}
    \item H. Lodish et al., \textit{Molecular Cell Biology}, 9th Edition, W. H. Freeman, 2021.
    \item B. Alberts et al., \textit{Molecular Biology of the Cell}, 7th Edition, W. W. Norton, 2022.
\end{itemize}

\newpage

\section{Microprocessor}
\subsection*{Course Code: 40513, Units: 3, Prerequisite: Computer Architecture, Co-requisite: None}

\textbf{Course Summary:} 
\begin{quote}
This course builds on previous courses in computer structure and architecture, teaching students how to design, implement, program, evaluate, and debug a computer system using a microprocessor or microcomputer chip along with peripheral chips. It also introduces advanced microprocessors and how to utilize their advanced features.
\end{quote}

\textbf{Course Outline:}
\begin{itemize}
    \item Introduction to the evolution and development of microprocessors
    \item Introduction to a fundamental and widely-used processor
    \item Minimum mode: 
    \begin{itemize}
        \item 16-bit memory architecture and internal structure of 8086
        \item Pin configuration and basic cycles
        \item Instruction set and assembly programming
        \item System architecture in minimum mode
        \item Interrupts
    \end{itemize}
    \item Maximum mode:
    \begin{itemize}
        \item Pin configuration and basic cycles
        \item System architecture in maximum mode
        \item 8284 and 8288 chips
    \end{itemize}
    \item Peripheral circuits:
    \begin{itemize}
        \item Timers, serial and parallel ports
        \item User interface (keyboard, display)
        \item Environmental interfaces (rotary encoder, DA and AD converters)
        \item Introduction to interrupt controller and DMA controller
    \end{itemize}
    \item 8087 coprocessor:
    \begin{itemize}
        \item Internal structure
        \item Instruction set
        \item Interaction with the main 8086 processor
    \end{itemize}
    \item Introduction to a microcontroller
    \item Instruction set overview
    \item Interrupt management
    \item Peripheral circuits introduction:
    \begin{itemize}
        \item Communication interfaces (I2C, SBI, etc.)
        \item PWM unit
        \item Various timers
        \item Different ports
        \item ADC and DAC units
    \end{itemize}
    \item Power management unit introduction
    \item Introduction to a graphics processor
    \item Brief overview of PCI-Express interface
    \item Summary of graphics processor architecture and programming
\end{itemize}

\textbf{Books Used:}
\begin{itemize}
    \item W. A. Triebel and A. Singh, \textit{The 8088 and 8086 Microprocessors}, Prentice-Hall, 2003.
    \item M. A. Mazidi, \textit{The 80x86 IBM PC \& Compatible Computers}, Volume II, Prentice Hall International Inc., 1995.
    \item W. A. Smith, \textit{ARM Microcontroller Interfacing: Hardware and Software}, Elektor International, 2010.
    \item M. A. Mazidi et al., \textit{The AVR Microcontroller and Embedded Systems: Using Assembly and C}, Prentice Hall, 2011.
    \item D. Kirk and W. M. Hwu, \textit{Programming Massively Parallel Processors}, Morgan Kaufmann, 2012.
\end{itemize}

\textbf{Evaluation:}
\begin{itemize}
    \item Theoretical exercises: 3 points
    \item Mid-term and final exams: 15 points
    \item Quizzes: 2 points
\end{itemize}

\newpage

\section{Hardware Lab}
\subsection*{Course Code: 40102, Units: 1, Prerequisite: Computer Architecture Lab, Co-requisite: None}

\textbf{Course Summary:} 
\begin{quote}
The aim of this laboratory course is to enhance students' skills in designing and implementing hardware systems for practical problems relevant nationally or globally. Applications include embedded systems, data acquisition and monitoring systems, Internet of Things, and digital systems in industrial and medical domains. Students are expected to apply their knowledge from hardware, computer architecture, operating systems, and system-level programming to build an efficient device for solving a real-world problem. The topics are project-based and may vary across semesters.
\end{quote}

\textbf{Course Outline:}
\begin{itemize}
    \item Implementation of a vital signs sampling and patient monitoring system using a mobile phone
    \item Design and simulation of a traffic control system
    \item Implementation of an access control system based on fingerprint/RFID card
    \item Design of vehicle positioning and parking spot status notification system in parking lots using cameras/proximity sensors/light sensors, etc.
    \item Visual matching detection of car body condition upon entry and exit using four cameras to check for damages during parking
    \item Implementation of at least one IoT-based or Cyber-Physical System application using sensors/actuators/modern boards
    \item Practical implementation of at least one hardware-in-the-loop experiment for signal processing or similar applications, preferably using Simulink/Matlab software
\end{itemize}

\textbf{Books Used:}
\begin{itemize}
    \item M. A. Mazidi, Sh. Chen, and E. Ghaemi, \textit{Atmel ARM Programming for Embedded Systems}, MicroDigitalEd, Vol 5, 2017.
    \item M. A. Mazidi, Se. Naimi, and Sa. Naimi, \textit{AVR Microcontroller and Embedded Systems: Using Assembly and C}, MicroDigitalEd, 2017.
    \item R. H. Chu and D. D. Lu, "Project-based lab learning teaching for power electronics and drives," \textit{IEEE Transactions on Education}, Vol. 51, No.1, pp. 108-113, 2008.
    \item J. Ma and J.V. Nickerson, "Hands-on, simulated, and remote laboratories: A comparative literature review," \textit{ACM Computing Surveys}, Vol. 38, No. 3, 2006.
\end{itemize}

\end{document}
